\documentclass[a4paper]{amsart}

%                  Trash .aux file after toggling
\usepackage{stmaryrd}
\usepackage{graphicx}
\usepackage[margin=1in]{geometry}
\usepackage[lutzsyntax]{virginialake}\aftrianglefalse
\usepackage[pdfborder={0 0 0}]{hyperref}
\usepackage[urw-garamond]{mathdesign}

%--------- Theorem etc
\newtheorem{thm}{Theorem}[section]
\newtheorem{cor}[thm]{Corollary}
\newtheorem{lem}[thm]{Lemma}
\newtheorem{pro}[thm]{Proposition}

\theoremstyle{remark}
\newtheorem{rem}[thm]{Remark}
\newtheorem{exa}[thm]{Example}

\theoremstyle{definition}
\newtheorem{defi}[thm]{Definition}
%---------

\begin{document}

\title[Normalisation Control in Deep Inference   via Atomic Flows II]
      {Normalisation Control in Deep Inference\\ via Atomic Flows II}

\author{Alessio Guglielmi and Tom Gundersen}
%\address{University of Bath, Bath BA2 7AY, UK}

\thanks{This work was in part funded by an Overseas Research Scholarship and a Research Studentship, both from the University of Bath, and by the British Council Alliance Programme.}

\keywords{Normalisation, deep inference, cut elimination, atomic flows}

\subjclass{F.4.1 Mathematical Logic---Proof theory}

% \begin{abstract}
% \end{abstract}

\maketitle

%===============================================================================
\section{Introduction}

%What?
We are interested in normalising derivations in systems of propositional logic. To us, normalisation is the elimination of cuts or, more generally, `dead ends' from derivations. An abstract depiction of a dead end is the following
\[
\atomicflow{
( 0  , 4  )*{\afaidnw{}{}};
( 6  , 3  )*{\afvjd6{}a};
( 0  , 2.3)*{\aflabelright{\bar a}};
( 2  , 2  )*{\afvj4};
(-2  , 1  )*{\afvju6a{}};
( 4  ,-2  )*{\afaiunw{}{}};
(-3.5, 0  )*{\invisiblemark};
( 7.5, 0  )*{\invisiblemark}}
\quad,
\]
where the atom $\bar a$ is first created by an axiom and then destroyed by a cut (represented by horizontal bars), hence never reaching the premiss nor the conclusion of the derivation.

%Why?
We believe that understanding normalisation is crucial for solving problems related to complexity and identity of proofs. To argue about these complicated notions a simple language is needed and in particular a language without bureaucracy.

%State of affairs
At the core of our work lies \emph{atomic flows} which were introduced in \cite{GuglGund:07:Normalis:lr}. Atomic flows are graphical representations of derivations which are largely syntax independent and bureaucracy free. It was shown how atomic flows are useful in defining new normal forms and in arguing about normalisation. However, the normalisation procedures presented employed several different techniques and were relatively complicated. To get a more intuitive understanding of the mechanisms at play, something more simplistic was desired.

%How?
Using the language of atomic flows we show in this paper how we can normalise derivations using a simple cut-copy-paste approach. We cut away some parts of a derivation to normalise it, at the expense of destroying it. However, by making many copies and cutting away different parts from each of them, it is possible to paste all the pieces together and create a normal derivation.

%Novelty
In \cite{GuglGund:07:Normalis:lr} our focus was on local transformations of derivations and how the atomic flow of a derivation was sufficient to guide normalisation. In this paper we take this further by considering global normalisation procedures. We show that information about the connectedness of axioms and cuts is all that is needed to guide normalisation.

%WOW!
The newfound simplicity allowed us to find a strongly normalising streamlining procedure.

\newcommand{\SKS}{\mathsf{SKS}}
%Constraints
The results in this paper are presented in the deep inference formalism the calculus of structures and system $\SKS$, but we strive at generality and as long as a system has atomic structural rules and linear logical rules (which is expected in deep inference), it should not be difficult to adapt our results.

%Outline
We start with a brief introduction to deep inference (Section~\ref{SectDeepInference}) and to atomic flows (Section~\ref{SectAtomicFlows}), before presenting our main result; strongly normalising streamlining (Section~\ref{SectStreamlining}).

%===============================================================================
\section{Background on Deep Inference}\label{SectDeepInference}

\newcommand{\fff}{\mathsf f}
\newcommand{\ttt}{\mathsf t}
\newcommand{\ot}{\mathbin\shortleftarrow}

% TODO: check that we only every use the variable names in the following definition

%---------------------------------------
\begin{defi}\label{DefFormulae}
\emph{Formulae}, $\alpha$, $\beta$, $\gamma$, $\delta$ are freely built from: \emph{units}, $\fff$ (false), $\ttt$ (true); \emph{atoms}, $a$, $b$, $c$, $d$; \emph{disjunction} and \emph{conjunction}, ${\vlsbr[\alpha.\beta]}$ and $\vlsbr(\alpha.\beta)$. On the set of atoms a (non-identical) involution $\bar\cdot$ is defined, and dual atom occurrences, as $a$ and $\bar a$, can appear in formulae. We denote \emph{contexts}, \emph{i.e.}, formulae with a hole, by $\xi\vlhole$ and $\zeta\vlhole$; we also use \emph{multiple} contexts, $\xi\vlhole\cdots\vlhole$, \emph{i.e.}, formulae with many holes; for example, if $\xi\{a\}$ is $\vls(b.[a.c])$, then $\xi\vlhole$ is $\vls(b.[\vlhole.c])$, $\xi\{b\}$ is $\vls(b.[b.c])$ and $\xi\vlscn(a.d)$ is $\vls(b.[(a.d).c])$; if $\xi\{a\}\{b\}\{c\}$ is $\vls(b.[(a.d).c])$ then $\xi\{b\}\{c\}\{a\}$ is $\vls(c.[(b.d).a])$.
\end{defi}

%---------------------------------------
\begin{rem}
Negation is only defined for atoms, which is not a limitation thanks to De Morgan laws.
\end{rem}

\newcommand{\one}{{\mathchoice{\scriptstyle\mathbf1}
                              {\scriptstyle\mathbf1}
                              {\scriptstyle\mathbf1}
                              {\scriptscriptstyle\mathbf1}}}
\newcommand{\two}{{\mathchoice{\scriptstyle\mathbf2}
                              {\scriptstyle\mathbf2}
                              {\scriptstyle\mathbf2}
                              {\scriptscriptstyle\mathbf2}}}
%---------------------------------------
\begin{defi}\label{DefDerivation}
\emph{Inference rules}, $\rho$, have one \emph{premiss} and one \emph{conclusion}, and their \emph{instances} are used in \emph{inference steps} to rewrite inside formulae. A \emph{derivation}, $\Phi$, from $\alpha$ (\emph{premiss}) to $\beta$ (\emph{conclusion}) is a chain of inference steps with $\alpha$ at the top and $\beta$ at the bottom, and is usually indicated by $\vlder{\Phi}{\mathcal S}{\beta}{\alpha}$, where $\mathcal S$ is the name of the deductive system or a set of inference rules; a \emph{proof} is a derivation from $\ttt$; besides $\Phi$, we denote derivations with $\Psi$. We denote with $\xi\{\Phi\}$ the result of including every formula of the derivation $\vlder{\Phi}{}{\beta}{\alpha}$ into the context $\xi\vlhole$. Since we adopt deep inference, $\vlder{\xi\{\Phi\}}{}{\xi\{\beta\}}{\xi\{\alpha\}}$ is a valid derivation. Furthermore, $\xi\left\{\vlder{}{}{\beta_1}{\alpha_1}\right\}\cdots\left\{\vlder{}{}{\beta_n}{\alpha_n}\right\}$ denotes
\[
\vlderivation
{
 \vlin{=}{}{\xi\{\beta_1\}\cdots\{\beta_{n-1}\}\left\{\vlder{}{}{\beta_n}{\alpha_n}\right\}}
 {
  \vlin{=}{}{\vdots}
  {
   \vlin{=}{}{\xi\{\beta_1\}\cdots\{\beta_{i-1}\}\left\{\vlder{}{}{\beta_i}{\alpha_i}\right\}\{\alpha_{i+1}\}\cdots\{\alpha_n\}}
   {
    \vlin{=}{}{\vdots}
    {
     \vlhy{\xi\left\{\vlder{}{}{\beta_1}{\alpha_1}\right\}\{\alpha_2\}\cdots\{\alpha_n\}}
    }
   }
  }
 }
}\quad.
\]
\end{defi}

Now we define the standard deductive system, $\SKS$, for classical propositional logic in deep inference, which is used throughout the paper \cite{Brun:03:Atomic-C:oz,Brun:06:Cut-Elim:cq,Brun:06:Locality:zh,BrunTiu:01:A-Local-:mz}.

\newcommand{\ai}{\mathsf{ai}}
\newcommand{\aw}{\mathsf{aw}}
\newcommand{\ac}{\mathsf{ac}}
\newcommand{\aid}{{\ai{\downarrow}}}
\newcommand{\awd}{{\aw{\downarrow}}}
\newcommand{\acd}{{\ac{\downarrow}}}
\newcommand{\aiu}{{\ai{\uparrow}}}
\newcommand{\awu}{{\aw{\uparrow}}}
\newcommand{\acu}{{\ac{\uparrow}}}
\newcommand{\swi}{\mathsf{s}}
\newcommand{\med}{\mathsf{m}}
%---------------------------------------
\begin{defi}
System $\SKS$ in the calculus of structures is defined by the following \emph{structural} rules:
\[
\begin{array}{@{}c@{}c@{}c@{}}
      \vlinf{\aid}{}{\vls[a.{\bar a}]}{\ttt}&
\qquad\vlinf{\awd}{}a\fff&
\qquad\vlinf{\acd}{}a{\vls[a.a]}\\
\noalign{\smallskip}
      \emph{interaction}&
\qquad\emph{weakening}&
\qquad\emph{contraction}\\
\noalign{\bigskip}
      \vlinf{\aiu}{}\fff{\vls(a.{\bar a})}&
\qquad\vlinf{\awu}{}\ttt a&
\qquad\vlinf{\acu}{}{\vls (a.a)}a\\
\noalign{\smallskip}
      \emph{cointeraction}&
\qquad\emph{coweakening}&
\qquad\emph{cocontraction}\\
\end{array}\quad,
\]
and by the two \emph{logical} rules:
\[
\begin{array}{@{}c@{}c@{}}
\vlinf{\swi}{}{\vls[(\alpha.\beta).\gamma]}{\vls(\alpha.[\beta.\gamma])}&\qquad
\vlinf{\med}{}{\vls([\alpha.\gamma].[\beta.\delta])}
              {\vls[(\alpha.\beta).(\gamma.\delta)]}\\
\noalign{\smallskip}
\emph{switch}&\qquad\emph{medial}\\
\end{array}\quad.
\]
The rule cointeraction is also called an (\emph{atomic}) \emph{cut}. In addition to the rules shown, there is a rule $\vldownsmash{\vlinf={}\delta\gamma}$, such that $\gamma$ and $\delta$ are opposite sides in one of the following equations:
\vlstore{
\vls[\alpha.\beta]         &=\vls[\beta.\alpha]         \quad,&
\vls[\alpha.\fff]          &=\vls[\alpha]               \quad,\\
\vls(\alpha.\beta)         &=\vls(\beta.\alpha)         \quad,&
\vls(\alpha.\ttt)          &=\vls(\alpha)               \quad,\\
\vls[[\alpha.\beta].\gamma]&=\vls[\alpha.[\beta.\gamma]]\quad,&
\vls[\ttt.\ttt]            &=\vls[\ttt]                 \quad,\\
\vls((\alpha.\beta).\gamma)&=\vls(\alpha.(\beta.\gamma))\quad,&
\vls(\fff.\fff)            &=\vls(\fff)                 \quad\vldot}
\begin{align*}
\vlread
\end{align*}
We do not always show the instances of rule $=$, and when we do show them, we gather several contiguous instances into one.
\end{defi}

%===============================================================================
\section{Background on Atomic Flows}\label{SectAtomicFlows}

\newbox\contrup\setbox\contrup=\hbox{$
   \divide\atflowunit by5\multiply\atflowunit by3\afsetunits
   \atomicflow{(0,0)*{\afacu{}{}{}{}{}{}}}$}
\newbox\contrdown\setbox\contrdown=\hbox{$
   \divide\atflowunit by5\multiply\atflowunit by3\afsetunits
   \atomicflow{(0,0)*{\afacd{}{}{}{}{}{}}}$}
\newbox\interdown\setbox\interdown=\hbox{$
   \divide\atflowunit by5\multiply\atflowunit by3\afsetunits
   \atomicflow{(0,0)*{\afaid{}{}{}{}{}{}}}$}
\newbox\interup\setbox\interup=\hbox{$
   \divide\atflowunit by5\multiply\atflowunit by3\afsetunits
   \atomicflow{(0,0)*{\afaiu{}{}{}{}{}{}}}$}
\newbox\weakdown\setbox\weakdown=\hbox{$
   \divide\atflowunit by5\multiply\atflowunit by3\afsetunits
   \atomicflow{(0,0)*{\afawd{}{}{}{}{}{}}}$}
\newbox\weakup\setbox\weakup=\hbox{$
   \divide\atflowunit by5\multiply\atflowunit by3\afsetunits
   \atomicflow{(0,0)*{\afawu{}{}{}{}{}{}}}$}
%===============================================================================
\section{Streamlining}\label{SectStreamlining}

Streamlining is an up-down symmetric generalisation of cut-elimination which applies to derivations as well as to proofs. The cut rule is not, in general, admissible from derivations, so when defining a normal form for derivations we cannot define it based on which rules it contains. We therefore define streamlining based on the shape of the atomic flow associated with a derivation. Intuitively a derivation is streamlined if its associated normal form contains no `dead ends', where a dead end is a path which does not start at the premiss nor end at the conclusion. We recall the definition from \cite{GuglGund:07:Normalis:lr}:

%---------------------------------------
\begin{defi}
An $\SKS$ derivation is \emph{streamlined} if, in its associated atomic flow, there are no paths from interaction or weakening vertices to cointeraction or coweakening vertices.
\end{defi}

%---------------------------------------
\begin{rem}\label{RemStr}
It immediately follows from the definition that the diagram below describes the shape of a streamlined derivation:
\[
\atomicflow{
(-10,11)*{\afvjm4};
%---
(-15, 5)*{\copy\contrup};
(-10, 5)*{\affr{28}8};
( -5, 5)*{\copy\contrdown};
( 10, 5)*{\copy\interdown};
( 10, 5)*{\affr88};
( 20, 5)*{\copy\weakdown};
( 20, 5)*{\affr88};
%---
(-20, 0)*{\afvjm2};
(-10, 0)*{\afvjm2};
(  0, 0)*{\afvjm2};
( 10, 0)*{\afvjm2};
( 20, 0)*{\afvjm2};
%---
(-20,-5)*{\copy\weakup};
(-20,-5)*{\affr88};
(-10,-5)*{\copy\interup};
(-10,-5)*{\affr88};
(  5,-5)*{\copy\contrup};
( 10,-5)*{\affr{28}8};
( 15,-5)*{\copy\contrdown};
%---
(  10,-11)*{\afvjm4};
}\quad.
\]
\end{rem}

From our point of view, the main challenge in streamlining a derivation is to make sure there are no paths from axioms to cuts. Once this is achieved it is straightforward, as shown in \cite{GuglGund:07:Normalis:lr}, to use (confluent and strongly normalising) weakening reductions to obtain a streamlined derivation. This motivates the following definition:

%---------------------------------------
\begin{defi}
An $\SKS$ derivation is \emph{weakly streamlined} if, in its associated atomic flow, there are no paths from interaction vertices to cointeraction vertices.
\end{defi}

%---------------------------------------
\begin{rem}\label{RemWeakStr}
It immediately follows from the definition that the diagram below describes the shape of a weakly streamlined derivation:
\[
\atomicflow{
(-5, 11)*{\afvjm4};
%---
( -5, 5)*{\affr{18}8};
(-10, 5)*{\copy\contrup};
( -5, 5)*{\copy\weakdown};
(  0, 5)*{\copy\contrdown};
( 10, 5)*{\affr88};
( 10, 5)*{\copy\interdown};
%---
(-10, 0)*{\afvjm2};
(  0, 0)*{\afvjm2};
( 10, 0)*{\afvjm2};
%---
(-10,-5)*{\affr88};
(-10,-5)*{\copy\interup};
(  5,-5)*{\affr{18}8};
(  0,-5)*{\copy\contrup};
(  5,-5)*{\copy\weakup};
( 10,-5)*{\copy\contrdown};
%---
(  5,-11)*{\afvjm4};
}\quad.
\]
\end{rem}

\subsection{The Core}

As a consequence of having atomic structural inference rules, deep inference allows us to trivially assume that all axioms occur in the premiss and all cuts occur in the conclusion of a derivation. Furthermore, we can (co)contract axioms and cuts so that every atom occurs in at most one axiom and at most one cut. If a derivation, $\Phi$, has this shape we can obtain a weakly streamlined derivation simply by considering all the atoms which occur in both an axiom and a cut and replacing the axiom and the cut by their conclusion and premiss respectively. This weakly streamlined derivation clearly does not have the same premiss and conclusion as $\Phi$, but, as will be made precise later, their atomic flows are almost the same and for this reason we call this the \emph{core of\/ $\Phi$}. The core will be the building block we will use to create a weakly streamlined derivation with the same premiss and conclusion as $\Phi$.

We first define the core of an atomic flow.

\newcommand{\Core}{\mathsf{Core}}

\begin{defi}\label{DefFlowCore}
Given an atomic flow
\[
A=\atomicflow
{
(-21, 8.5)*{\afvjm{9}};
(-13, 8)*{\afaidm{}{}{}{}{}{}};
( -5, 8.5)*{\afvjm{9}};
(-18, 0)*{\affr{8}{8}};
(-16, 2)*{a_1};
( -8, 0)*{\affr{8}{8}};
( -6, 2)*{\bar a_1};
( -5,-8.5)*{\afvjm{9}};
(-13,-8)*{\afaium{}{}{}{}{}{}};
(-21,-8.5)*{\afvjm{9}};
%------------
(0,0)*{\cdots};
%------------
(21, 8.5)*{\afvjm{9}};
(13, 8)*{\afaidm{}{}{}{}{}{}};
( 5, 8.5)*{\afvjm{9}};
( 8, 0)*{\affr{8}{8}};
(10, 2)*{a_n};
(18, 0)*{\affr{8}{8}};
(20, 2)*{\bar a_n};
( 5,-8.5)*{\afvjm{9}};
(13,-8)*{\afaium{}{}{}{}{}{}};
(21,-8.5)*{\afvjm{9}};
%------------
(33, 11)*{\afvjm4};
%---
( 33, 5)*{\affr{18}8};
( 28, 5)*{\copy\contrup};
( 33, 5)*{\copy\weakdown};
( 38, 5)*{\copy\contrdown};
( 48, 5)*{\affr88};
( 48, 5)*{\copy\interdown};
%---
( 28, 0)*{\afvjm2};
( 38, 0)*{\afvjm2};
( 48, 0)*{\afvjm2};
%---
( 28,-5)*{\affr88};
( 28,-5)*{\copy\interup};
( 43,-5)*{\affr{18}8};
( 38,-5)*{\copy\contrup};
( 43,-5)*{\copy\weakup};
( 48,-5)*{\copy\contrdown};
%---
( 43,-11)*{\afvjm4};
}\quad,
\]
the \emph{core of $A$} is defined to be
\[
\Core(A)=
\atomicflow
{
(-21,10)*{\afvjm{12}};
(-17,15)*{\afvj2};
(-17,10)*{\affr{6}{8}};
(-17,10)*{\copy\contrup};
(-17, 5)*{\afvjm{2}};
(-18, 0)*{\affr{8}{8}};
(-16, 2)*{a_1};
(-21,-10)*{\afvjm{12}};
(-17,-15)*{\afvj2};
(-17,-10)*{\affr{6}{8}};
(-17,-10)*{\copy\contrdown};
(-17, -5)*{\afvjm{2}};
%
( -9,15)*{\afvj2};
( -9,10)*{\affr{6}{8}};
( -9,10)*{\copy\contrup};
( -9, 5)*{\afvjm{2}};
( -5,10)*{\afvjm{12}};
( -8, 0)*{\affr{8}{8}};
( -6, 2)*{\bar a_1};
( -9,-15)*{\afvj2};
( -9,-10)*{\affr{6}{8}};
( -9,-10)*{\copy\contrdown};
( -9, -5)*{\afvjm{2}};
( -5,-10)*{\afvjm{12}};
( -8, 0)*{\affr{8}{8}};
%------------
(0,0)*{\cdots};
%------------
( 9,15)*{\afvj2};
( 9,10)*{\affr{6}{8}};
( 9,10)*{\copy\contrup};
( 9, 5)*{\afvjm{2}};
( 5,10)*{\afvjm{12}};
( 8, 0)*{\affr{8}{8}};
(10, 2)*{a_n};
( 9,-15)*{\afvj2};
( 9,-10)*{\affr{6}{8}};
( 9,-10)*{\copy\contrdown};
( 9, -5)*{\afvjm{2}};
( 5,-10)*{\afvjm{12}};
( 8, 0)*{\affr{8}{8}};
%
(21,10)*{\afvjm{12}};
(17,15)*{\afvj2};
(17,10)*{\affr{6}{8}};
(17,10)*{\copy\contrup};
(17, 5)*{\afvjm{2}};
(18, 0)*{\affr{8}{8}};
(20, 2)*{\bar a_n};
(21,-10)*{\afvjm{12}};
(17,-15)*{\afvj2};
(17,-10)*{\affr{6}{8}};
(17,-10)*{\copy\contrdown};
(17, -5)*{\afvjm{2}};
%---------
(33, 12.5)*{\afvjm7};
%---
( 33, 5)*{\affr{18}8};
( 28, 5)*{\copy\contrup};
( 33, 5)*{\copy\weakdown};
( 38, 5)*{\copy\contrdown};
( 48, 5)*{\affr88};
( 48, 5)*{\copy\interdown};
%---
( 28, 0)*{\afvjm2};
( 38, 0)*{\afvjm2};
( 48, 0)*{\afvjm2};
%---
( 28,-5)*{\affr88};
( 28,-5)*{\copy\interup};
( 43,-5)*{\affr{18}8};
( 38,-5)*{\copy\contrup};
( 43,-5)*{\copy\weakup};
( 48,-5)*{\copy\contrdown};
%---
( 43,-12.5)*{\afvjm7};
}\quad,
\]
where the subflow of $A$ labelled $a_i$ (resp., $\bar a_i$) is isomorphic to the sublfow of $\Core(A)$ labelled $a_i$ (resp., $\bar a_i$) for every $1\leq i\leq n$ and the rightmost subflow of $A$ is isomorphic to the rightmost sublfow of $\Core(A)$.
\end{defi}

\begin{defi}\label{DefCore}
Given a derivation $\vlder{\Phi}{}{\beta}{\alpha}$ with associated atomic flow $A$, a \emph{core of\/ $\Phi$} is defined to be a derivation $\vlder{\Core(\Phi)}{}{\vls[\beta.(a_n.{\bar a_n}).\cdots.(a_1.{\bar a_1})]}{\vls([a_1.{\bar a_1}].\cdots.[a_n.{\bar a_n}].\alpha)}$ with associated atomic flow $\Core(A)$.
\end{defi}

We finally state a theorem whose proof can be found in the full version of this paper.

\begin{thm}
Given any derivation $\vlder{\Phi}{}{\beta}{\alpha}$, $\Core(\Phi)$ exists.
\end{thm}

\subsection{The Normaliser}

We now show how to build a weakly streamlined derivation. We present a `skeleton' with black boxes where we can plug the core of the derivation we want to normalise. The novelty of this method is that no knowledge of the shape of the core is required. The only information needed to build the normaliser is the number of atoms which must be eliminated from the premiss and conclusion of the core in order to get the premiss and conclusion of the original derivation. The procedure is strongly normalising, but not confluent. Two kinds of choices have to be made, the order in which the atoms are eliminated and for each atom, $a$, which comes first, $a$ or $\bar a$. It is crucial to observe that the atomic flows contain all the information needed for making these choices.

\newcommand{\contr}{\mathsf{c}}
\newcommand{\cod}{{\contr{\downarrow}}}
\newcommand{\cou}{{\contr{\uparrow}}}

\newcommand{\Norm}{\mathsf{Norm}}

\begin{defi}
The \emph{normaliser}, $\Norm(\Phi,a_1,\dots,a_n)$, is an operator taking as input a sequence of atoms and a derivation of the form
\[
\vlder{\Phi}{}{\vls[\beta.(a_n.{\bar a_n}).\cdots.(a_1.{\bar a_1})]}{\vls([a_1.{\bar a_1}].\cdots.[a_n.{\bar a_n}].\alpha)}\quad,
\]
where $\alpha$ and $\beta$ are formulae and returning a derivation of the form
\[
\vlder{\Norm(\Phi,a_1,\dots,a_n)}{}{\beta}{\alpha}\quad.
\]

We define $\Norm$ inductively on the number of arguments. Let $\Norm(\Phi)=\Phi$ and for $n>0$ let $\Norm(\Phi,a_1,\dots,a_n)$ be
\newbox\DeltaTopK
\setbox\DeltaTopK=
\hbox{$
\vlderivation
{
 \vlde{\Norm(\Phi,a_1,\dots,a_{n-1})}{}{\vls[\beta.(\vlinf{\awu}{}{\ttt}{a_n}.\bar a_n)]}
 {
  \vlhy{\vls(\vlinf{\aid}{}{\vls[a_n.\bar a_n]}{\ttt}.\alpha)}
 }
}$
}
\newbox\DeltaBotK
\setbox\DeltaBotK=
\hbox{
$\vlderivation
{
 \vlde{\Norm(\Phi,a_1,\dots,a_{n-1})}{}{\vls[\beta.\vlinf{\aiu}{}{\fff}{\vls(a_n.\bar a_n)}]}
 {
  \vlhy{\vls([a_n.\vlinf{\awd}{}{\bar a_n}{\fff}].\alpha)}
 }
}$
}
\newbox\DeltaK
\setbox\DeltaK=
\hbox{$
\vlderivation
{
 \vlde{\Norm(\Phi,a_1,\dots,a_{n-1})}{}{\vls[\beta.(a_n.\vlinf{\awu}{}{\ttt}{\bar a_n})]}
 {
  \vlhy{\vls([\vlinf{\awd}{}{a_n}{\fff}.\bar a_n].\alpha)}
 }
}$
}
\[
\vlderivation
{
 \vlin{\cod}{}{\beta}
 {
  \vlin{\swi}{}{\vls[\vlinf{\cod}{}{\beta}{\vls[\beta.\beta]}.\box\DeltaBotK]}
  {
   \vlin{\swi}{}{\vls([\beta.\box\DeltaK].\alpha)}
   {
    \vlin{\cou}{}{\vls(\box\DeltaTopK.\vlinf{\cou}{}{\vls(\alpha.\alpha)}{\alpha})}
    {
     \vlhy{\alpha}
    }
   }
  }
 }
}\quad.
\]
\end{defi}

\begin{pro}\label{PropFlowNorm}
If the atomic flow of
\[\vlder{\Norm(\Phi,a_1,\dots,a_{n-1})}{}{\vls[\beta.(a_n.\bar a_n)]}{\vls([a_n.\bar a_n].\alpha)}
\quad\mbox{is}\quad
\atomicflow
{
(-8, 6)*{\afvjm{4}};
(-2, 6)*{\afvju{4}{a_n}{}};
( 2, 6)*{\afvju{4}{}{\bar a_n}};
( 8, 6)*{\afvjm{4}};
(-5, 0)*{\affr{8}{8}};
(-3, 2)*{A};
( 5, 0)*{\affr{8}{8}};
( 7, 2)*{B};
( 8,-6)*{\afvjm{4}};
(-2,-6)*{\afvjd{4}{a_n}{}};
( 2,-6)*{\afvjd{4}{}{\bar a_n}};
(-8,-6)*{\afvjm{4}};
}\quad,
\]
then the atomic flow of
\[
\vlder{\Norm(\Phi,a_1,\dots,a_n)}{}{\beta}{\alpha}
\quad\mbox{is}\quad
\atomicflow
{
% cocontractions
%  outer
(-13.5,36.5)*{\afacumexsqcol{}{}{}{}{}{}{33}{4}{}{Green}{Green}};
(  2.5,36.5)*{\afacumexsqcol{}{}{}{}{}{}{33}{4}{}{Green}{Green}};
%  inner
( -8, 13)*{\afvjmcol{18}{Green}};
( 14,  0)*{\afvjmcol{44}{Green}};
(  3, 26)*{\afacumnwexsqcol{}{}{}{}{11}{2}{Green}{Green}};
(  8, 13)*{\afvjm{18}};
( 30, 0)*{\afvjmcol{44}{Green}};
( 19, 26)*{\afacumnwexsqcol{}{}{}{}{11}{2}{}{Green}};
% top boxes
(-22, 34)*{\afaidcol{}{}{}{}{}{}{Red}{Red}};
(-27, 26)*{\affr{8}{8}};
(-25, 28)*{A_1};
(-17, 26)*{\affr{8}{8}};
(-15, 28)*{B_1};
(-24, 18)*{\afawucol{}{}{}{}{}{Red}};
( -9, 13)*{\afcjlcol{22}{18}{Red}};
% middle boxes
( -2,  8)*{\afawdcol{}{}{}{}{}{Green}};
(-5, 0)*{\affr{8}{8}};
(-3, 2)*{A_2};
( 5, 0)*{\affr{8}{8}};
( 7, 2)*{B_2};
(  2, -8)*{\afawucol{}{}{}{}{}{Red}};
% bottom boxes
( 22,-34)*{\afaiucol{}{}{}{}{}{}{Green}{Green}};
( 17,-26)*{\affr{8}{8}};
( 19,-24)*{A_3};
( 27,-26)*{\affr{8}{8}};
( 29,-24)*{B_3};
( 24,-18)*{\afawdcol{}{}{}{}{}{Green}};
(  9,-13)*{\afcjlcol{22}{18}{Green}};
% contractions
%  inner
( -8,-12.75)*{\afvjm{17.5}};
(-30,0.25)*{\afvjmcol{43.5}{Red}};
(-19,-27.5)*{\afacdmnwexsqcol{}{}{}{}{11}{2}{Red}{}};
(  8,-12.75)*{\afvjmcol{17.5}{Red}};
(-14,0.25)*{\afvjmcol{43.5}{Red}};
( -3,-27.5)*{\afacdmnwexsqcol{}{}{}{}{11}{2}{Red}{Red}};
%  outer
( 13.5,-36)*{\afacdmexsqcol{}{}{}{}{}{}{33}{4}{Red}{}{Red}};
( -2.5,-36)*{\afacdmexsqcol{}{}{}{}{}{}{33}{4}{Red}{}{Red}};
}\quad,
\]
where $A_1,A_2,A_3$ are isomorphic to $A$ and $B_1,B_2,B_3$ are isomorphic to $B$ and all the edges that might be in paths from the evidenced interaction vertex are colored in red and all the edges that might be in paths from the evidenced cointeraction vertex are colored in green.
\end{pro}

We now state our main theorem whose proof can be found in the full version of this paper.

\begin{thm}
Given a derivation $\vlder{\Phi}{}{\beta}{\alpha}$, there are atoms $a_1,\dots,a_n$ in $\Phi$ such that $\vlder{\Norm(\Core(\Phi),a_1,\dots,a_n)}{}{\beta}{\alpha}$ is weakly streamlined.
\end{thm}

%===========================================
\section{Conclusion}

\bibliographystyle{alpha}
\bibliography{di-biblio}

\end{document}