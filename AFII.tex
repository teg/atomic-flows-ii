\documentclass[a4paper]{llncs}

\usepackage[lutzsyntax]{virginialake}\aftrianglefalse
\usepackage[urw-garamond]{mathdesign}
\usepackage{hyperref}

\begin{document}

\title{Normalisation Control in Deep Inference\\ via Atomic Flows II}

\author{Alessio Guglielmi\inst{1,2} and Tom Gundersen\inst{1}}
\institute{University of Bath, Bath BA2 7AY, UK \and INRIA, Nancy-Grand Est, France}

\thanks{Guglielmi is supported by EPSRC grant EP/E042805/1 \emph{Complexity and Non-determinism in Deep Inference} and by an ANR \emph{Senior Chaire d'Excellence} titled \emph{Identity and Geometric Essence of Proofs}. Gundersen is supported by an \emph{Overseas Research Scholarship} and a \emph{Research Studentship} both of the University of Bath.}

\maketitle

%===============================================================================
\section{Introduction}

We are interested in the normalisation of deep-inference derivations in propositional classical logic. Cuts are admissible from proofs and, dually, axioms are admissible from contradictions. However, neither are admissible from derivations. We therefore work with \emph{streamlining}, introduced in \cite{GuglGund:07:Normalis:lr}, a new, symmetric, notion of normalisation, which generalises both cut and axiom elimination.

At the core of our work lie \emph{atomic flows}, which were introduced in \cite{GuglGund:07:Normalis:lr}. Atomic flows are graphs, similar to Buss flow graphs \cite{Buss:91:The-Unde:uq} and proof nets \cite{Gira:87:Linear-L:wm}, obtained from derivations by tracing their atom occurrences and forgetting everything except how atoms are created, copied, contracted and destroyed. Atomic flows are largely syntax independent and bureaucracy free (in the sense of Girard \cite{Gira:89:Geometry:sh}). We showed how atomic flows are useful in defining new normal forms for derivations and in arguing about normalisation.

In particular, streamlining was defined based on atomic flows. Intuitively, a derivation is streamlined if every path in the associated atomic flow can be extended to reach the top or the bottom of the flow. Seen from the point of view of derivations it means that if you pick an atom occurrence from a streamlined derivation you can trace it up to the premiss or down to the conclusion. Since a proof has no atoms in its premiss (only the unit `true'), tracing atom occurrences upwards from a cut can not lead to the premiss. Hence, a streamlined proof is cut free.

In this paper, we present a new streamlining procedure. Similarly to our previous result the procedure is based on a particular way of gluing together pieces of derivation, which is possible due to the symmetries of deep inference. However, the novelties are that no transformation of the original derivation is needed before the pieces are glued together, and much less information about the atomic flow associated with the derivation is used to guide the procedure. In particular, we only need to know which axioms are connected with which cuts. Since we use less information, all the axioms and cuts we are eliminating are indistinguishable, so, unlike the previous procedures, no strategy for streamlining is necessary.

Contrary to what one might expect from cut elimination, the complexity of the procedure is not determined by the number of cuts being eliminated. The complexity is $O(2^n)$, where $n$ is the number of atoms which occur in at least one of the cuts we eliminate. In particular, every step of the procedure eliminate all the cuts where a given atom occurs, making termination a triviality.

Also somewhat surprising is the fact that if we consider the original derivation as a `black box', then the streamlined derivation only depends on the number of atoms being eliminated, not on the contents of the `black box'. We can therefore present our normalisation procedure as building a `skeleton', for a given number of atoms, in which we can plug derivations in order to streamline them.

\newcommand{\SKS}{\mathsf{SKS}}
The results in this paper are presented in the deep-inference formalism the calculus of structures \cite{Gugl:06:A-System:kl}, in particular system $\SKS$ \cite{BrunTiu:01:A-Local-:mz,Brun:04:Deep-Inf:rq}, but we strive at generality and it should not be difficult to adapt our results to any deep-inference formalism and any propositional system as long as we have atomic structural rules and linear logical rules (something we always expect in deep inference and something which is not achievable elsewhere).

In the future, we believe it will be possible to improve on these results by extending them to modal \cite{Brun:07:Deep-Seq:fk,HeinStew:05:Purity-T:tg,StewStou:05:A-System:tg,Stou:06:A-Deep-I:rt} and first order \cite{Brun:04:Deep-Inf:rq,Brun:06:Cut-Elim:cq} logics and by making the procedure quasipolynomial, along the lines of a quasipolynomial cut-elimination procedure we are working on.

%===============================================================================
\section{Background on Deep Inference}
\newcommand{\fff}{\mathsf f}
\newcommand{\ttt}{\mathsf t}
\newcommand{\ot}{\mathbin\shortleftarrow}


\newcommand{\one}{{\mathchoice{\scriptstyle\mathbf1}
                              {\scriptstyle\mathbf1}
                              {\scriptstyle\mathbf1}
                              {\scriptscriptstyle\mathbf1}}}
\newcommand{\two}{{\mathchoice{\scriptstyle\mathbf2}
                              {\scriptstyle\mathbf2}
                              {\scriptstyle\mathbf2}
                              {\scriptscriptstyle\mathbf2}}}
%---------------------------------------
\begin{definition}\label{DefDerivation}
\emph{Inference rules}, $\rho$, have one \emph{premiss} and one \emph{conclusion}, and their \emph{instances} are used in \emph{inference steps} to rewrite inside formulae. A \emph{derivation}, $\Phi$, from $\alpha$ (\emph{premiss}) to $\beta$ (\emph{conclusion}) is a chain of inference steps with $\alpha$ at the top and $\beta$ at the bottom, and is usually indicated by $\vlder{\Phi}{\mathcal S}{\beta}{\alpha}$, where $\mathcal S$ is the name of the deductive system or a set of inference rules; a \emph{proof} is a derivation from $\ttt$; besides $\Phi$, we denote derivations with $\Psi$. We denote with $\xi\{\Phi\}$ the result of including every formula of the derivation $\vlder{\Phi}{}{\beta}{\alpha}$ into the context $\xi\vlhole$. Since we adopt deep inference, $\vlder{\xi\{\Phi\}}{}{\xi\{\beta\}}{\xi\{\alpha\}}$ is a valid derivation. Furthermore, $\xi\left\{\vlder{}{}{\beta_1}{\alpha_1}\right\}\cdots\left\{\vlder{}{}{\beta_n}{\alpha_n}\right\}$ denotes
\[
\vlderivation
{
 \vlin{=}{}{\xi\{\beta_1\}\cdots\{\beta_{n-1}\}\left\{\vlder{}{}{\beta_n}{\alpha_n}\right\}}
 {
  \vlin{=}{}{\vdots}
  {
   \vlin{=}{}{\xi\{\beta_1\}\cdots\{\beta_{i-1}\}\left\{\vlder{}{}{\beta_i}{\alpha_i}\right\}\{\alpha_{i+1}\}\cdots\{\alpha_n\}}
   {
    \vlin{=}{}{\vdots}
    {
     \vlhy{\xi\left\{\vlder{}{}{\beta_1}{\alpha_1}\right\}\{\alpha_2\}\cdots\{\alpha_n\}}
    }
   }
  }
 }
}\quad.
\]
\end{definition}

\newcommand{\KS}{\mathsf{KS}}
Now we define the two standard deductive systems for classical propositional logic in deep inference that are used throughout the paper. $\KS$ is analytic, in the sense that premisses only contain subformulae of conclusions, and $\SKS$ is not \cite{Brun:03:Atomic-C:oz,Brun:06:Cut-Elim:cq,Brun:06:Locality:zh,BrunTiu:01:A-Local-:mz}.

\newcommand{\ai}{\mathsf{ai}}
\newcommand{\aw}{\mathsf{aw}}
\newcommand{\ac}{\mathsf{ac}}
\newcommand{\aid}{{\ai{\downarrow}}}
\newcommand{\awd}{{\aw{\downarrow}}}
\newcommand{\acd}{{\ac{\downarrow}}}
\newcommand{\aiu}{{\ai{\uparrow}}}
\newcommand{\awu}{{\aw{\uparrow}}}
\newcommand{\acu}{{\ac{\uparrow}}}
\newcommand{\swi}{\mathsf{s}}
\newcommand{\med}{\mathsf{m}}
%---------------------------------------
\begin{definition}
System $\SKS$ in the calculus of structures is defined by the following \emph{structural} rules:
\[
\begin{array}{@{}c@{}c@{}c@{}}
      \vlinf{\aid}{}{\vls[a.{\bar a}]}{\ttt}&
\qquad\vlinf{\awd}{}a\fff&
\qquad\vlinf{\acd}{}a{\vls[a.a]}\\
\noalign{\smallskip}
      \emph{interaction}&
\qquad\emph{weakening}&
\qquad\emph{contraction}\\
\noalign{\bigskip}
      \vlinf{\aiu}{}\fff{\vls(a.{\bar a})}&
\qquad\vlinf{\awu}{}\ttt a&
\qquad\vlinf{\acu}{}{\vls (a.a)}a\\
\noalign{\smallskip}
      \emph{cointeraction}&
\qquad\emph{coweakening}&
\qquad\emph{cocontraction}\\
\end{array}\quad,
\]
and by the two \emph{logical} rules:
\[
\begin{array}{@{}c@{}c@{}}
\vlinf{\swi}{}{\vls[(\alpha.\beta).\gamma]}{\vls(\alpha.[\beta.\gamma])}&\qquad
\vlinf{\med}{}{\vls([\alpha.\gamma].[\beta.\delta])}
              {\vls[(\alpha.\beta).(\gamma.\delta)]}\\
\noalign{\smallskip}
\emph{switch}&\qquad\emph{medial}\\
\end{array}\quad.
\]
The rule cointeraction is also called an (\emph{atomic}) \emph{cut}. In addition to the rules shown, there is a rule $\vldownsmash{\vlinf={}\delta\gamma}$, such that $\gamma$ and $\delta$ are opposite sides in one of the following equations:
%TODO: make this alignment work with the LNCS style
\vlstore{
\vls[\alpha.\beta]         &=\vls[\beta.\alpha]         \quad,&
\vls[\alpha.\fff]          &=\vls[\alpha]               \quad,\\
\vls(\alpha.\beta)         &=\vls(\beta.\alpha)         \quad,&
\vls(\alpha.\ttt)          &=\vls(\alpha)               \quad,\\
\vls[[\alpha.\beta].\gamma]&=\vls[\alpha.[\beta.\gamma]]\quad,&
\vls[\ttt.\ttt]            &=\vls[\ttt]                 \quad,\\
\vls((\alpha.\beta).\gamma)&=\vls(\alpha.(\beta.\gamma))\quad,&
\vls(\fff.\fff)            &=\vls(\fff)                 \quad\vldot}
%\begin{align}
%\vlread
%\end{align}
We do not always show the instances of rule $=$, and when we do show them, we gather several contiguous instances into one. System $\KS$ is the same as $\SKS$, but without the rules $\aiu$, $\awu$ and $\acu$. A \emph{cut-free} derivation is a derivation where $\aiu$ is not used. All derivations in this paper are in $\SKS$, unless indicated otherwise.
\end{definition}

\begin{remark}
The representations of $\SKS$ derivations in this paper are sometimes ambiguous. This is not a problem, as the derivations themselves are well defined and we can always get a more detailed representation should we need to, for instance when extracting atomic flows from derivations. The ambiguity arises because we omit or collapse equations and because we do not distinguish the different equation rules. In particular we need to know if
\[
\vlinf{=}{}{\vls((a.a).a)}{\vls(a.(a.a))}\quad,
\]
is an application of associativity $\left(\vlinf{=}{}{\vls((a.b).c)}{\vls(a.(b.c))}\right)$ or commutativity $\left(\vlinf{=}{}{\vls((b.c).a)}{\vls(a.(b.c))}\right)$ to be able to map atom occurrences in the premiss to atom occurrences in the conclusion.
\end{remark}

% TODO: find the right place for this example

\begin{example}\label{ExaFormalismA}
Consider the $\SKS$ derivation
\[
\vls[\vlinf{\acd}{}{a}{\vls[a.a]}.\vlinf{\aiu}{}{\fff}{\vls(\vlinf{\acd}{}{b}{\vls[b.b]}.\bar b)}]\quad,
\]
which is a shorthand for
\[
\vlderivation
{
 \vlin{\aiu}{}{a}
 {
  \vlin{\acd}{}{\vls[a.(b.\bar b)]}
  {
   \vlin{\acd}{}{\vls[a.([b.b].\bar b)]}
   {
    \vlhy{\vls[a.a.([b.b].\bar b)]}
   }
  }
 }
}.
\]
It is implicit in Definition~\ref{DefFormulae} that the ordering of holes in a formula context with multiple holes is given. This decides the order in which we `sequentialise' the short-hand representation of a derivation in Definition~\ref{DefDerivation}. In this example we used a left-to-right ordering, but we could just as well have used a right-to-left ordering:
\[
\vlderivation
{
 \vlin{\acd}{}{a}
 {
  \vlin{\aiu}{}{\vls[a.a]}
  {
   \vlin{\acd}{}{\vls[a.a.(b.\bar b)]}
   {
    \vlhy{\vls[a.a.([b.b].\bar b)]}
   }
  }
 }
}.
\]
In Formalism A, a new deep-inference formalism currently under development, the above shorthand is a representative of the equivalence class containing both the left-to-right and the right-to-left orderings, in addition to the intermediate:
\[
\vlderivation
{
 \vlin{\aiu}{}{a}
 {
  \vlin{\acd}{}{\vls[a.(b.\bar b)]}
  {
   \vlin{\acd}{}{\vls[a.a.(b.\bar b)]}
   {
    \vlhy{\vls[a.a.([b.b].\bar b)]}
   }
  }
 }
}.
\]
For the purposes of this paper, which of these three derivations we use is inessential, as the atomic flows associated with each of them are are the same. For simplicity we let the shorthand above refer to one unique derivation, but we could just as well have worked in Formalism A.
\end{example}
%===============================================================================

\section{Background on Atomic Flows}\label{SectAtomicFlows}

\newbox\contrup\setbox\contrup=\hbox{$
   \divide\atflowunit by5\multiply\atflowunit by3\afsetunits
   \atomicflow{(0,0)*{\afacu{}{}{}{}{}{}}}$}
\newbox\contrdown\setbox\contrdown=\hbox{$
   \divide\atflowunit by5\multiply\atflowunit by3\afsetunits
   \atomicflow{(0,0)*{\afacd{}{}{}{}{}{}}}$}
\newbox\interdown\setbox\interdown=\hbox{$
   \divide\atflowunit by5\multiply\atflowunit by3\afsetunits
   \atomicflow{(0,0)*{\afaid{}{}{}{}{}{}}}$}
\newbox\interup\setbox\interup=\hbox{$
   \divide\atflowunit by5\multiply\atflowunit by3\afsetunits
   \atomicflow{(0,0)*{\afaiu{}{}{}{}{}{}}}$}
\newbox\weakdown\setbox\weakdown=\hbox{$
   \divide\atflowunit by5\multiply\atflowunit by3\afsetunits
   \atomicflow{(0,0)*{\afawd{}{}{}{}{}{}}}$}
\newbox\weakup\setbox\weakup=\hbox{$
   \divide\atflowunit by5\multiply\atflowunit by3\afsetunits
   \atomicflow{(0,0)*{\afawu{}{}{}{}{}{}}}$}
%===============================================================================
\section{Streamlining}\label{SectStreamlining}

%TODO: make sure we know what atomic flows are

Streamlining is an up-down symmetric generalisation of cut-elimination, which applies to derivations as well as to proofs. The cut is not admissible from derivations, so we define a streamlined derivation to be a derivation with a specific shape, and in the special case when the derivation has no premiss, \emph{i.e.}, it is a proof, it is cut-free. Intuitively a derivation is streamlined if its associated atomic flow contains no `dead ends', where a dead end is a path which does not start at the premiss nor end at the conclusion. We recall the definition from \cite{GuglGund:07:Normalis:lr}:

%TODO: what does the derivation mean

%---------------------------------------
\begin{definition}
An $\SKS$ derivation is \emph{streamlined} if its associated atomic flow can be represented as follows:
\[
\atomicflow{
(-10,11)*{\afvjm4};
%---
(-15, 5)*{\copy\contrup};
(-10, 5)*{\affr{28}8};
( -5, 5)*{\copy\contrdown};
( 10, 5)*{\copy\interdown};
( 10, 5)*{\affr88};
( 20, 5)*{\copy\weakdown};
( 20, 5)*{\affr88};
%---
(-20, 0)*{\afvjm2};
(-10, 0)*{\afvjm2};
(  0, 0)*{\afvjm2};
( 10, 0)*{\afvjm2};
( 20, 0)*{\afvjm2};
%---
(-20,-5)*{\copy\weakup};
(-20,-5)*{\affr88};
(-10,-5)*{\copy\interup};
(-10,-5)*{\affr88};
(  5,-5)*{\copy\contrup};
( 10,-5)*{\affr{28}8};
( 15,-5)*{\copy\contrdown};
%---
(  10,-11)*{\afvjm4};
}\quad.
\]
\end{definition}

%TODO: Why is cut elimination a special case of streamlining?
%TODO: what is a path?

The main challenge in streamlining a derivation is to make sure there are no paths from axioms to cuts. Once this is achieved it is straightforward, as shown in \cite{GuglGund:07:Normalis:lr}, to use (confluent and strongly normalising) weakening reductions to obtain a streamlined derivation. This motivates the following definition:

%---------------------------------------
\begin{definition}
An $\SKS$ derivation is \emph{weakly streamlined} if its associated atomic flow can be represented as follows:
\[
\atomicflow{
(-5, 11)*{\afvjm4};
%---
( -5, 5)*{\affr{18}8};
(-10, 5)*{\copy\contrup};
( -5, 5)*{\copy\weakdown};
(  0, 5)*{\copy\contrdown};
( 10, 5)*{\affr88};
( 10, 5)*{\copy\interdown};
%---
(-10, 0)*{\afvjm2};
(  0, 0)*{\afvjm2};
( 10, 0)*{\afvjm2};
%---
(-10,-5)*{\affr88};
(-10,-5)*{\copy\interup};
(  5,-5)*{\affr{18}8};
(  0,-5)*{\copy\contrup};
(  5,-5)*{\copy\weakup};
( 10,-5)*{\copy\contrdown};
%---
(  5,-11)*{\afvjm4};
}\quad.
\]
\end{definition}


%TODO: structure of the section

\subsection{The Core}

The \emph{core of a derivation} is obtained by removing each pair of axiom and cut instances that is connected by a path in the atomic flow associated to the derivation. As can be seen in Definition~\ref{DefFlowCore} below this leaves most of the atomic flow associated with the original derivation untouched. Furthermore the core of a derivation is clearly weakly streamlined.

The core of a derivation is the basic building block we will use when creating a weakly streamlined derivation in the next subsection. Note that when creating the core the only information needed is whether two axiom and cut instances are connected or not. We do not need to study or manipulate the atomic flow any further. This is in contrast to the local procedures which were employed in \cite{GuglGund:07:Normalis:lr}.

We first define what the core of an atomic flow is and then the definition of the core of a derivation will follow.

\newcommand{\Core}{\mathsf{Core}}

%TODO: edges are about to fall off, the space is not even
%TODO: I'm not sure using atoms to label flows is a good idea, because, given an atom, one would expect a canonical flow to correspond to it.
%TODO: nobody can understand this definition

\begin{definition}\label{DefFlowCore}
Given an atomic flow
\[
A=\atomicflow
{
(-21, 8.5)*{\afvjm{9}};
(-13, 8)*{\afaidm{}{}{}{}{}{}};
( -5, 8.5)*{\afvjm{9}};
(-18, 0)*{\affr{8}{8}};
(-16, 2)*{a_1};
( -8, 0)*{\affr{8}{8}};
( -6, 2)*{\bar a_1};
( -5,-8.5)*{\afvjm{9}};
(-13,-8)*{\afaium{}{}{}{}{}{}};
(-21,-8.5)*{\afvjm{9}};
%------------
(0,0)*{\cdots};
%------------
(21, 8.5)*{\afvjm{9}};
(13, 8)*{\afaidm{}{}{}{}{}{}};
( 5, 8.5)*{\afvjm{9}};
( 8, 0)*{\affr{8}{8}};
(10, 2)*{a_n};
(18, 0)*{\affr{8}{8}};
(20, 2)*{\bar a_n};
( 5,-8.5)*{\afvjm{9}};
(13,-8)*{\afaium{}{}{}{}{}{}};
(21,-8.5)*{\afvjm{9}};
%------------
(33, 11)*{\afvjm4};
%---
( 33, 5)*{\affr{18}8};
( 28, 5)*{\copy\contrup};
( 33, 5)*{\copy\weakdown};
( 38, 5)*{\copy\contrdown};
( 48, 5)*{\affr88};
( 48, 5)*{\copy\interdown};
%---
( 28, 0)*{\afvjm2};
( 38, 0)*{\afvjm2};
( 48, 0)*{\afvjm2};
%---
( 28,-5)*{\affr88};
( 28,-5)*{\copy\interup};
( 43,-5)*{\affr{18}8};
( 38,-5)*{\copy\contrup};
( 43,-5)*{\copy\weakup};
( 48,-5)*{\copy\contrdown};
%---
( 43,-11)*{\afvjm4};
}\quad,
\]
the \emph{core of $A$} is defined to be
\[
\Core(A)=
\atomicflow
{
(-21,10)*{\afvjm{12}};
(-17,15)*{\afvj2};
(-17,10)*{\affr{6}{8}};
(-17,10)*{\copy\contrup};
(-17, 5)*{\afvjm{2}};
(-18, 0)*{\affr{8}{8}};
(-16, 2)*{a_1};
(-21,-10)*{\afvjm{12}};
(-17,-15)*{\afvj2};
(-17,-10)*{\affr{6}{8}};
(-17,-10)*{\copy\contrdown};
(-17, -5)*{\afvjm{2}};
%
( -9,15)*{\afvj2};
( -9,10)*{\affr{6}{8}};
( -9,10)*{\copy\contrup};
( -9, 5)*{\afvjm{2}};
( -5,10)*{\afvjm{12}};
( -8, 0)*{\affr{8}{8}};
( -6, 2)*{\bar a_1};
( -9,-15)*{\afvj2};
( -9,-10)*{\affr{6}{8}};
( -9,-10)*{\copy\contrdown};
( -9, -5)*{\afvjm{2}};
( -5,-10)*{\afvjm{12}};
( -8, 0)*{\affr{8}{8}};
%------------
(0,0)*{\cdots};
%------------
( 9,15)*{\afvj2};
( 9,10)*{\affr{6}{8}};
( 9,10)*{\copy\contrup};
( 9, 5)*{\afvjm{2}};
( 5,10)*{\afvjm{12}};
( 8, 0)*{\affr{8}{8}};
(10, 2)*{a_n};
( 9,-15)*{\afvj2};
( 9,-10)*{\affr{6}{8}};
( 9,-10)*{\copy\contrdown};
( 9, -5)*{\afvjm{2}};
( 5,-10)*{\afvjm{12}};
( 8, 0)*{\affr{8}{8}};
%
(21,10)*{\afvjm{12}};
(17,15)*{\afvj2};
(17,10)*{\affr{6}{8}};
(17,10)*{\copy\contrup};
(17, 5)*{\afvjm{2}};
(18, 0)*{\affr{8}{8}};
(20, 2)*{\bar a_n};
(21,-10)*{\afvjm{12}};
(17,-15)*{\afvj2};
(17,-10)*{\affr{6}{8}};
(17,-10)*{\copy\contrdown};
(17, -5)*{\afvjm{2}};
%---------
(33, 12.5)*{\afvjm7};
%---
( 33, 5)*{\affr{18}8};
( 28, 5)*{\copy\contrup};
( 33, 5)*{\copy\weakdown};
( 38, 5)*{\copy\contrdown};
( 48, 5)*{\affr88};
( 48, 5)*{\copy\interdown};
%---
( 28, 0)*{\afvjm2};
( 38, 0)*{\afvjm2};
( 48, 0)*{\afvjm2};
%---
( 28,-5)*{\affr88};
( 28,-5)*{\copy\interup};
( 43,-5)*{\affr{18}8};
( 38,-5)*{\copy\contrup};
( 43,-5)*{\copy\weakup};
( 48,-5)*{\copy\contrdown};
%---
( 43,-12.5)*{\afvjm7};
}\quad,
\]
where the subflow of $A$ labelled $a_i$ (resp., $\bar a_i$) is isomorphic to the subflow of $\Core(A)$ labelled $a_i$ (resp., $\bar a_i$) for every $1\leq i\leq n$ and the rightmost subflow of $A$ is isomorphic to the rightmost sublfow of $\Core(A)$.
\end{definition}

We now define the core of a derivation. The core is weakly streamlined at the expense of new atoms in the premiss and conclusion. In the next section we will see how we can glue several cores together to eliminate these atoms, but still obtain a weakly streamlined derivation.

%TODO: at this point the reader is lost, what are you trying to convey?

\begin{definition}\label{DefCore}
Given a derivation $\vlder{\Phi}{}{\beta}{\alpha}$ with associated atomic flow $A$, a \emph{core of\/ $\Phi$} is defined to be a derivation $\vlder{\Core(\Phi)}{}{\vls[\beta.(a_n.{\bar a_n}).\cdots.(a_1.{\bar a_1})]}{\vls([a_1.{\bar a_1}].\cdots.[a_n.{\bar a_n}].\alpha)}$ with associated atomic flow $\Core(A)$.
\end{definition}

To obtain the core of a derivation is straightforward and a detailed proof can be found in the full version of this paper.

\begin{proposition}\label{PropStreamlinedCore}
Any $\Core(\Phi)$ is weakly streamlined.
\end{proposition}

\begin{lemma}\label{LemSuperSwitch}
Given a context $\xi\vlhole$ and a formula $\alpha$ there exist derivations $\vlder{}{\{\swi\}}{\xi\{\alpha\}}{\vls(\alpha.\xi\{\ttt\})}$ and $\vlder{}{\{\swi\}}{\vls[\xi\{\fff\}.\alpha]}{\xi\{\alpha\}}$.
\end{lemma}

\begin{proof}
We show how to construct the first derivation, the second one can be done by symmetry. We argue by induction on the number of atoms in $\xi\vlhole$. The base case, $\xi\vlhole=\vlhole$, is trivial and the inductive cases are:

\[
\vlderivation
{
 \vlin{=}{}{\xi\{\alpha\}}
 {
  \vlin{\swi}{}{\vls[\vlder{\Psi}{\{\swi\}}{\xi'\{\alpha\}}{\vls(\alpha.\xi'\{\ttt\})}.\beta]}
  {
   \vlin{=}{}{\vls(\alpha.[\xi'\{\ttt\}.\beta])}
   {
    \vlhy{\vls(\alpha.\xi\{\ttt\})}
   }
  }
 }
}\qquad\mbox{and}\qquad
\vlderivation
{
 \vlin{=}{}{\xi\{\alpha\}}
 {
  \vlin{=}{}{\vls(\vlder{\Psi'}{\{\swi\}}{\xi'\{\alpha\}}{\vls(\alpha.\xi'\{\ttt\})}.\beta)}
  {
   \vlhy{\vls(\alpha.\xi\{\ttt\})}
  }
 }
}\quad,
\]
for some $\xi'\vlhole$ and $\beta$ where $\beta$ is not a unit and $\Psi$ and $\Psi'$ exist by the inductive hypothesis.
\end{proof}

\begin{lemma}\label{LemDecompInt}
Given a derivation $\vlder{}{}{\beta}{\alpha}$ with associated atomic flow $A$, there exists a derivation
\[
\vlder{}{\SKS\setminus\{\aid,\aiu\}}{\vls[\beta.\vlinf{\aiu}{}{\fff}{\vls(b_m.\bar b_m)}.\cdots.\vlinf{\aiu}{}{\fff}{\vls(b_1.\bar b_1)}]}{\vls(\vlinf{\aid}{}{\vls[a_1.\bar a_1]}{\ttt}.\cdots.\vlinf{\aid}{}{\vls[a_n.\bar a_n]}{\ttt}.\alpha)}
\]
with associated atomic flow $A$, for some atoms $a_1,\dots,a_n,b_1,\dots,b_m$.
\end{lemma}

\begin{proof}
Using Lemma~\ref{LemSuperSwitch} apply the following transformations to each of the (co)interaction instances in $\Phi$:
\[
\vlderivation
{
 \vlde{\Psi'}{}{\delta}
 {
  \vlin{\aid}{}{\xi\vlsbr[a.{\bar a}]}
  {
   \vlde{\Psi}{}{\xi\{\ttt\}}
   {
    \vlhy{\gamma}
   }
  }
 }
}\quad\rightarrow\quad
\vlderivation
{
 \vlde{\Psi'}{}{\delta}
 {
  \vlde{}{\{\swi\}}{\xi\vlsbr[a.{\bar a}]}
  {
   \vlhy{\vlsbr(\vlinf{\aid}{}{\vls[a.{\bar a}]}{\ttt}.\vlder{\Psi}{}{\xi\{\ttt\}}{\gamma})}
  }
 }
}\qquad\mbox{and}\qquad
\vlderivation
{
 \vlde{\Psi'}{}{\delta}
 {
  \vlin{\aid}{}{\xi\{\fff\}}
  {
   \vlde{\Psi}{}{\xi\vlsbr(a.{\bar a})}
   {
    \vlhy{\gamma}
   }
  }
 }
}\quad\rightarrow\quad
\vlderivation
{
 \vlde{}{\{\swi\}}{\vlsbr[\vlder{\Psi'}{}{\delta}{\xi\{\fff\}}.\vlinf{\aiu}{}{\fff}{\vls(a.{\bar a})}]}
 {
  \vlde{\Psi}{}{\xi\vlsbr(a.{\bar a})}
  {
   \vlhy{\gamma}
  }
 }
}\quad.
\]
\end{proof}

\begin{lemma}\label{LemGenericContraction}
Given a formula $\alpha$ and a positive integer $n$, there exist derivations $\vlder{}{\{\acd,\med\}}{\alpha}{\bigvee_{i=1}^{n}\alpha}$ and $\vlder{}{\{\acu,\med\}}{\bigwedge_{i=1}^{n}\alpha}{\alpha}$.
\end{lemma}

\begin{theorem}
Given a derivation $\vlder{\Phi}{}{\beta}{\alpha}$, the derivation $\Core(\Phi)$ exists.
\end{theorem}

\begin{proof}
We build $\Core(\Phi)$ as follows:
\[
\vlderivation
{
 \vlde{\Psi_2}{\{\acd,\med\}}{\vls[\beta.(a_n.{\bar a_n}).\cdots.(a_1.{\bar a_1})]}
 {
  \vlde{\Phi'}{\SKS\setminus\{\aid,\aiu\}}{{\vls[\beta.\vlinf{\aiu}{}{\fff}{\vls(c_l.\bar c_l)}.\cdots.\vlinf{\aiu}{}{\fff}{\vls(c_1.\bar c_1)}.(a_n.\bar a_n).\cdots.(a_n.\bar a_n).\cdots.(a_1.\bar a_1).\cdots.(a_1.\bar a_1)]}}
  {
   \vlde{\Psi_1}{\{\acu,\med\}}{{\vls([a_1.\bar a_1].\cdots.[a_1.\bar a_1].\cdots.[a_n.\bar a_n].\cdots.[a_n.\bar a_n].\vlinf{\aid}{}{\vls[b_1.\bar b_1]}{\ttt}.\cdots.\vlinf{\aid}{}{\vls[b_k.\bar b_k]}{\ttt}.\alpha)}}
   {
    \vlhy{\vls([a_1.{\bar a_1}].\cdots.[a_n.{\bar a_n}].\alpha)}
   }
  }
 }
}\quad,
\]
where $a_1,\dots,a_n$ are distinct and pairwise non-dual atoms and there are no atoms in common between $b_1,\dots,b_k$ and their duals and $c_1,\dots,c_l$ and their duals, $\Phi'$ exists by Lemma~\ref{LemDecompInt} and $\Psi_1$ and $\Psi_2$ exist by Lemma~\ref{LemGenericContraction}.
\end{proof}


\subsection{The Normaliser}

%TODO: Still, obtaining the core requires knowledge of the original derivation, so, it's not quite clear what is the moral point you are trying to make.

We now show how to build a weakly streamlined derivation. We present the \emph{normaliser}, which is a `skeleton' where we can plug copies of the core of the derivation we want to normalise. The novelty of this method is that no knowledge of the shape of the core is required. The only information needed to build the normaliser is the number of atoms which must be eliminated from the premiss and conclusion of the core in order to obtain the desired derivation.

\newcommand{\contr}{\mathsf{c}}
\newcommand{\cod}{{\contr{\downarrow}}}
\newcommand{\cou}{{\contr{\uparrow}}}

\begin{remark}
In the non-atomic version of system $\SKS$ the derivations shown in Lemma~\ref{LemGenericContraction} correspond to repeated applications of (co)contractions. For this reason we sometimes write the inference rules $\vlinf{\cod}{}{\alpha}{\vls[\alpha.\alpha]}$ and $\vlinf{\cou}{}{\vls(\alpha.\alpha)}{\alpha}$ instead of the derivations $\vlder{}{\{\acd,\med\}}{\alpha}{\vls[\alpha.\alpha]}$ and $\vlder{}{\{\acu,\med\}}{\vls(\alpha.\alpha)}{\alpha}$.
\end{remark}

\newcommand{\Norm}{\mathsf{Norm}}
%TODO: the normaliser takes a derivation so it does depend on the core, contradict what was just said
\begin{definition}
For every $n\geq 0$ the \emph{normaliser}, $\Norm(\Phi,a_1,\dots,a_n)$, is an operator taking as input a sequence of atoms and a derivation of the form
\[
\vlder{\Phi}{}{\vls[\beta.(a_n.{\bar a_n}).\cdots.(a_1.{\bar a_1})]}{\vls([a_1.{\bar a_1}].\cdots.[a_n.{\bar a_n}].\alpha)}\quad,
\]
where $\alpha$ and $\beta$ are formulae and returning a derivation of the form
\[
\vlder{\Norm(\Phi,a_1,\dots,a_n)}{}{\beta}{\alpha}\quad.
\]

We define $\Norm$ inductively on the number of arguments. Let $\Norm(\Phi)=\Phi$ and for $n>0$ let $\Norm(\Phi,a_1,\dots,a_n)$ be
\newbox\DeltaTopK
\setbox\DeltaTopK=
\hbox{$
\vlderivation
{
 \vlde{\Norm(\Phi,a_1,\dots,a_{n-1})}{}{\vls[\beta.(\vlinf{\awu}{}{\ttt}{a_n}.\bar a_n)]}
 {
  \vlhy{\vls(\vlinf{\aid}{}{\vls[a_n.\bar a_n]}{\ttt}.\alpha)}
 }
}$
}
\newbox\DeltaBotK
\setbox\DeltaBotK=
\hbox{
$\vlderivation
{
 \vlde{\Norm(\Phi,a_1,\dots,a_{n-1})}{}{\vls[\beta.\vlinf{\aiu}{}{\fff}{\vls(a_n.\bar a_n)}]}
 {
  \vlhy{\vls([a_n.\vlinf{\awd}{}{\bar a_n}{\fff}].\alpha)}
 }
}$
}
\newbox\DeltaK
\setbox\DeltaK=
\hbox{$
\vlderivation
{
 \vlde{\Norm(\Phi,a_1,\dots,a_{n-1})}{}{\vls[\beta.(a_n.\vlinf{\awu}{}{\ttt}{\bar a_n})]}
 {
  \vlhy{\vls([\vlinf{\awd}{}{a_n}{\fff}.\bar a_n].\alpha)}
 }
}$
}
\[
\vlderivation
{
 \vlin{}{}{\beta}
 {
  \vlin{\swi}{}{\vls[\vlinf{}{}{\beta}{\vls[\beta.\beta]}.\box\DeltaBotK]}
  {
   \vlin{\swi}{}{\vls([\beta.\box\DeltaK].\alpha)}
   {
    \vlin{}{}{\vls(\box\DeltaTopK.\vlinf{}{}{\vls(\alpha.\alpha)}{\alpha})}
    {
     \vlhy{\alpha}
    }
   }
  }
 }
}\quad.
\]
\end{definition}

\begin{remark}\label{RemFlowNorm}
If the atomic flow of $\vlder{\Norm(\Phi,a_1,\dots,a_{n-1})}{}{\vls[\beta.(a_n.\bar a_n)]}{\vls([a_n.\bar a_n].\alpha)}$ is
\[
\atomicflow
{
(-8, 6)*{\afvjm{4}};
(-2, 6)*{\afvju{4}{a_n}{}};
( 2, 6)*{\afvju{4}{}{\bar a_n}};
( 8, 6)*{\afvjm{4}};
(-5, 0)*{\affr{8}{8}};
(-3, 2)*{A};
( 5, 0)*{\affr{8}{8}};
( 7, 2)*{B};
( 8,-6)*{\afvjm{4}};
(-2,-6)*{\afvjd{4}{a_n}{}};
( 2,-6)*{\afvjd{4}{}{\bar a_n}};
(-8,-6)*{\afvjm{4}};
}\quad,
\]
then the atomic flow of $\vlder{\Norm(\Phi,a_1,\dots,a_n)}{}{\beta}{\alpha}$ is
\[
\atomicflow
{
% cocontractions
%  outer
(-13.5,36.5)*{\afacumexsqcol{}{}{}{}{}{}{33}{4}{}{Green}{Green}};
(  2.5,36.5)*{\afacumexsqcol{}{}{}{}{}{}{33}{4}{}{Green}{Green}};
%  inner
( -8, 13)*{\afvjmcol{18}{Green}};
( 14,  0)*{\afvjmcol{44}{Green}};
(  3, 26)*{\afacumnwexsqcol{}{}{}{}{11}{2}{Green}{Green}};
(  8, 13)*{\afvjm{18}};
( 30, 0)*{\afvjmcol{44}{Green}};
( 19, 26)*{\afacumnwexsqcol{}{}{}{}{11}{2}{}{Green}};
% top boxes
(-22, 34)*{\afaidcol{}{}{}{}{}{}{Red}{Red}};
(-27, 26)*{\affr{8}{8}};
(-25, 28)*{A_1};
(-17, 26)*{\affr{8}{8}};
(-15, 28)*{B_1};
(-24, 18)*{\afawucol{}{}{}{}{}{Red}};
( -9, 13)*{\afcjlcol{22}{18}{Red}};
% middle boxes
( -2,  8)*{\afawdcol{}{}{}{}{}{Green}};
(-5, 0)*{\affr{8}{8}};
(-3, 2)*{A_2};
( 5, 0)*{\affr{8}{8}};
( 7, 2)*{B_2};
(  2, -8)*{\afawucol{}{}{}{}{}{Red}};
% bottom boxes
( 22,-34)*{\afaiucol{}{}{}{}{}{}{Green}{Green}};
( 17,-26)*{\affr{8}{8}};
( 19,-24)*{A_3};
( 27,-26)*{\affr{8}{8}};
( 29,-24)*{B_3};
( 24,-18)*{\afawdcol{}{}{}{}{}{Green}};
(  9,-13)*{\afcjlcol{22}{18}{Green}};
% contractions
%  inner
( -8,-12.75)*{\afvjm{17.5}};
(-30,0.25)*{\afvjmcol{43.5}{Red}};
(-19,-27.5)*{\afacdmnwexsqcol{}{}{}{}{11}{2}{Red}{}};
(  8,-12.75)*{\afvjmcol{17.5}{Red}};		
(-14,0.25)*{\afvjmcol{43.5}{Red}};
( -3,-27.5)*{\afacdmnwexsqcol{}{}{}{}{11}{2}{Red}{Red}};
%  outer
( 13.5,-36)*{\afacdmexsqcol{}{}{}{}{}{}{33}{4}{Red}{}{Red}};
( -2.5,-36)*{\afacdmexsqcol{}{}{}{}{}{}{33}{4}{Red}{}{Red}};
}\quad,
\]
where $A_1$, $A_2$, $A_3$ are isomorphic to $A$ and $B_1$, $B_2$, $B_3$ are isomorphic to $B$ and all the edges that might be in paths from the evidenced interaction vertex are colored in red and all the edges that might be in paths from the evidenced cointeraction vertex are colored in green.
\end{remark}

By studying Definition~\ref{DefFlowCore} and Proposition~\ref{RemFlowNorm} we can see how a weakly streamlined derivation can be built. The proof is given in the full version of this paper.

\begin{lemma}\label{LemStreamlinedNorm}
Given a derivation $\Phi$ containing the atoms $a_1,\dots,a_n$ such that\/ $\Norm(\Phi,a_1,\dots,a_{k-1})$ and\/ $\Norm(\Phi,a_1,\dots,a_k)$ exist for some $0\leq k\leq n$, if\/ $\Norm(\Phi,a_1,\dots,a_{k-1})$ is weakly streamlined and none of the atoms $a_k,\dots,a_n$ occur in interaction nor cointeraction instances in\/ $\Norm(\Phi,a_1,\dots,a_{k-1})$ then
\begin{itemize}
 \item none of the atoms $a_{k+1},\dots,a_n$ occur in interaction nor cointeraction instances in $\Norm(\Phi,a_1,\dots,a_k)$ and
 \item $\Norm(\Phi,a_1,\dots,a_n)$ is weakly streamlined.
\end{itemize}
\end{lemma}

\begin{proof}
Refer to the atomic flow in Proposition~\ref{PropFlowNorm} to observe the following:
\begin{itemize}
 \item The only atoms which occur in (co)interaction instances in $\Norm(\Phi,a_1,\dots,a_k)$ which did not in $\Norm(\Phi,a_1,\dots,a_{k-1})$ are $a_k$ and $\bar a_k$.
 \item The only interaction and cointeraction nodes mapped to by $a_k$ in the atomic flow associated with $\Norm(\Phi,a_1,\dots,a_k)$ are the once that are evidenced and since the red and the green edges never coincide there can be no path between them. Furthermore, since $\Norm(\Phi,a_1,\dots,a_{k-1})$ is weakly streamlined a path from an interaction to a cointeraction vertex in the atomic flow associated with $\Norm(\Phi,a_1,\dots,a_k)$ must contain the red edge between $B_1$ and $B_2$ or the green edge between $A_2$ and $A_3$. However, $a_k$ and $\bar a_k$ map to these edges so there can be no path from an interaction to a cointeraction vertex.
\end{itemize}
\end{proof}

\begin{theorem}
Given a derivation $\vlder{\Phi}{}{\beta}{\alpha}$, there are atoms $a_1,\dots,a_n$ in $\Phi$ such that $\vlder{\Norm(\Core(\Phi),a_1,\dots,a_n)}{}{\beta}{\alpha}$ is weakly streamlined.
\end{theorem}

\begin{proof}
Choose $a_1,\dots,a_n$ and their duals such that no other atom maps to a path from an interaction to a cointeraction node in the atomic flow associated with $\Phi$, then the result follows by Proposition~\ref{PropStreamlinedCore} and Lemma~\ref{LemStreamlinedNorm}.
\end{proof}

%===========================================
\section{Conclusion}

\bibliographystyle{alpha}
\bibliography{biblio}

\end{document}