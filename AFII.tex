\documentclass[a4paper]{llncs}

\usepackage[lutzsyntax]{virginialake}\aftrianglefalse
%\usepackage[urw-garamond]{mathdesign}


\begin{document}

\title{Normalisation Control in Deep Inference\\ via Atomic Flows II}

\author{Alessio Guglielmi\inst{1,2} \and Tom Gundersen\inst{1}}
\institute{University of Bath, Bath BA2 7AY, UK \and INRIA, 615, rue du Jardin Botanique, 54603 Villers-les-Nancy Cedex, France}

\thanks{This work was in part funded by an Overseas Research Scholarship and a Research Studentship, both from the University of Bath, and by the British Council Alliance Programme.}

\maketitle

%===============================================================================
\section{Introduction}

%TODO: The introduction should have a better underlying scheme (what do you want to convey here?) and should contain some moral statements about what we do here.

%What?
We are interested in normalising derivations in propositional logic. To us, normalisation is the elimination of `unreachable' parts of derivations. An abstract depiction of an example of an unreachable part of a derivation is the following
\[
\atomicflow{
( 0  , 4  )*{\afaidnw{}{}};
( 6  , 3  )*{\afvjd6{}a};
( 0  , 2.3)*{\aflabelright{\bar a}};
( 2  , 2  )*{\afvj4};
(-2  , 1  )*{\afvju6a{}};
( 4  ,-2  )*{\afaiunw{}{}};
(-3.5, 0  )*{\invisiblemark};
( 7.5, 0  )*{\invisiblemark}}
\quad,
\]
where the atom $\bar a$ is first created by an axiom and then destroyed by a cut (represented by horizontal bars), hence never reaching the premiss nor the conclusion of the derivation.

%TODO: why, which problems, too generic
%TODO: language for what
%Why?
%We believe that understanding normalisation is crucial for solving problems related to complexity and identity of proofs. To argue about these complicated notions a simple language is needed and in particular a language without bureaucracy.

%State of affairs
At the core of our work lies \emph{atomic flows}, which were introduced in \cite{GuglGund:07:Normalis:lr}. Atomic flows are graphs obtained by removing all logical information from derivations and only retaining the structural part. They are largely syntax independent and bureaucracy free. It was shown how atomic flows are useful in defining new normal forms and in arguing about normalisation.

In particular, \emph{streamlining} was defined based on atomic flows. Streamlining is an up-down symmetric generalisation of cut-elimination, which applies to derivations as well as to proofs. Intuitively a derivation is streamlined if every path in the associated atomic flow can be extended to reach the top or the bottom of the atomic flow. Seen from the point of view of derivations it means that if you pick an atom occurrence from a streamlined derivation and start tracing it both up and down, you must reach either the premiss or the conclusion. Since a proof has no atoms in its premiss (only the unit `true') it is immediate that tracing atom occurrences upwards from a cut can not lead to the premiss, so a streamlined proof is cut free.

A streamlining procedure was presented combining local and global transformations to streamline derivations.

In this paper we present a new streamlining procedure based on a global transformations. The moral idea is that the only information needed to guide streamlining is information about the connectedness of axioms and cuts. Furthermore, no transformation of the original derivation is required except for removing the connected axioms and cuts. Due to the symmetries of deep inference, we are able to paste together copies of the original derivation, with some axioms and cuts removed, to obtain a derivation in normal form.

The novelty of this method is that the complexity of the procedure is decided by the number of atoms which occur in axioms and cuts that are eliminated, as opposed to the number of cuts. Furthermore, given the number of atoms being eliminated we can build a `skeleton' where we can plug the original derivation to obtain a streamlined derivation. This skeleton only depends on the number of atoms and not on the structure of the derivation being streamlined. As was also true of our previous results, logical inference rules and logical connectives do not play a role, which goes contrary to the tradition of cut elimination.

\newcommand{\SKS}{\mathsf{SKS}}
%Constraints
The results in this paper are presented in the deep inference formalism the calculus of structures \cite{Gugl:06:A-System:kl} and system $\SKS$ \cite{BrunTiu:01:A-Local-:mz}, but we strive at generality and it should not be difficult to adapt our results to any deep inference formalism and any propositional system as long as we have atomic structural rules and linear logical rules (something we always expect in deep inference).

For the full version of this paper please refer to \url{http://www.jklm.no/teg/AFII.pdf}
%===============================================================================

\newcommand{\fff}{\mathsf f}
\newcommand{\ttt}{\mathsf t}
\newcommand{\ot}{\mathbin\shortleftarrow}


\newcommand{\one}{{\mathchoice{\scriptstyle\mathbf1}
                              {\scriptstyle\mathbf1}
                              {\scriptstyle\mathbf1}
                              {\scriptscriptstyle\mathbf1}}}
\newcommand{\two}{{\mathchoice{\scriptstyle\mathbf2}
                              {\scriptstyle\mathbf2}
                              {\scriptstyle\mathbf2}
                              {\scriptscriptstyle\mathbf2}}}

\newcommand{\ai}{\mathsf{ai}}
\newcommand{\aw}{\mathsf{aw}}
\newcommand{\ac}{\mathsf{ac}}
\newcommand{\aid}{{\ai{\downarrow}}}
\newcommand{\awd}{{\aw{\downarrow}}}
\newcommand{\acd}{{\ac{\downarrow}}}
\newcommand{\aiu}{{\ai{\uparrow}}}
\newcommand{\awu}{{\aw{\uparrow}}}
\newcommand{\acu}{{\ac{\uparrow}}}
\newcommand{\swi}{\mathsf{s}}
\newcommand{\med}{\mathsf{m}}
%===============================================================================

\newbox\contrup\setbox\contrup=\hbox{$
   \divide\atflowunit by5\multiply\atflowunit by3\afsetunits
   \atomicflow{(0,0)*{\afacu{}{}{}{}{}{}}}$}
\newbox\contrdown\setbox\contrdown=\hbox{$
   \divide\atflowunit by5\multiply\atflowunit by3\afsetunits
   \atomicflow{(0,0)*{\afacd{}{}{}{}{}{}}}$}
\newbox\interdown\setbox\interdown=\hbox{$
   \divide\atflowunit by5\multiply\atflowunit by3\afsetunits
   \atomicflow{(0,0)*{\afaid{}{}{}{}{}{}}}$}
\newbox\interup\setbox\interup=\hbox{$
   \divide\atflowunit by5\multiply\atflowunit by3\afsetunits
   \atomicflow{(0,0)*{\afaiu{}{}{}{}{}{}}}$}
\newbox\weakdown\setbox\weakdown=\hbox{$
   \divide\atflowunit by5\multiply\atflowunit by3\afsetunits
   \atomicflow{(0,0)*{\afawd{}{}{}{}{}{}}}$}
\newbox\weakup\setbox\weakup=\hbox{$
   \divide\atflowunit by5\multiply\atflowunit by3\afsetunits
   \atomicflow{(0,0)*{\afawu{}{}{}{}{}{}}}$}
%===============================================================================
\section{Streamlining}\label{SectStreamlining}

%TODO: make sure we know what atomic flows are

Streamlining is an up-down symmetric generalisation of cut-elimination, which applies to derivations as well as to proofs. The cut is not admissible from derivations, so we define a streamlined derivation to be a derivation with a specific shape, and in the special case when the derivation has no premiss, \emph{i.e.}, it is a proof, it is cut-free. Intuitively a derivation is streamlined if its associated atomic flow contains no `dead ends', where a dead end is a path which does not start at the premiss nor end at the conclusion. We recall the definition from \cite{GuglGund:07:Normalis:lr}:

%TODO: what does the derivation mean

%---------------------------------------
\begin{definition}
An $\SKS$ derivation is \emph{streamlined} if its associated atomic flow can be represented as follows:
\[
\atomicflow{
(-10,11)*{\afvjm4};
%---
(-15, 5)*{\copy\contrup};
(-10, 5)*{\affr{28}8};
( -5, 5)*{\copy\contrdown};
( 10, 5)*{\copy\interdown};
( 10, 5)*{\affr88};
( 20, 5)*{\copy\weakdown};
( 20, 5)*{\affr88};
%---
(-20, 0)*{\afvjm2};
(-10, 0)*{\afvjm2};
(  0, 0)*{\afvjm2};
( 10, 0)*{\afvjm2};
( 20, 0)*{\afvjm2};
%---
(-20,-5)*{\copy\weakup};
(-20,-5)*{\affr88};
(-10,-5)*{\copy\interup};
(-10,-5)*{\affr88};
(  5,-5)*{\copy\contrup};
( 10,-5)*{\affr{28}8};
( 15,-5)*{\copy\contrdown};
%---
(  10,-11)*{\afvjm4};
}\quad.
\]
\end{definition}

%TODO: Why is cut elimination a special case of streamlining?
%TODO: what is a path?

The main challenge in streamlining a derivation is to make sure there are no paths from axioms to cuts. Once this is achieved it is straightforward, as shown in \cite{GuglGund:07:Normalis:lr}, to use (confluent and strongly normalising) weakening reductions to obtain a streamlined derivation. This motivates the following definition:

%---------------------------------------
\begin{definition}
An $\SKS$ derivation is \emph{weakly streamlined} if its associated atomic flow can be represented as follows:
\[
\atomicflow{
(-5, 11)*{\afvjm4};
%---
( -5, 5)*{\affr{18}8};
(-10, 5)*{\copy\contrup};
( -5, 5)*{\copy\weakdown};
(  0, 5)*{\copy\contrdown};
( 10, 5)*{\affr88};
( 10, 5)*{\copy\interdown};
%---
(-10, 0)*{\afvjm2};
(  0, 0)*{\afvjm2};
( 10, 0)*{\afvjm2};
%---
(-10,-5)*{\affr88};
(-10,-5)*{\copy\interup};
(  5,-5)*{\affr{18}8};
(  0,-5)*{\copy\contrup};
(  5,-5)*{\copy\weakup};
( 10,-5)*{\copy\contrdown};
%---
(  5,-11)*{\afvjm4};
}\quad.
\]
\end{definition}


%TODO: structure of the section

\subsection{The Core}

The \emph{core of a derivation} is obtained by removing each pair of axiom and cut instances that is connected by a path in the atomic flow associated to the derivation. As can be seen in Definition~\ref{DefFlowCore} below this leaves most of the atomic flow associated with the original derivation untouched. Furthermore the core of a derivation is clearly weakly streamlined.

The core of a derivation is the basic building block we will use when creating a weakly streamlined derivation in the next subsection. Note that when creating the core the only information needed is whether two axiom and cut instances are connected or not. We do not need to study or manipulate the atomic flow any further. This is in contrast to the local procedures which were employed in \cite{GuglGund:07:Normalis:lr}.

We first define what the core of an atomic flow is and then the definition of the core of a derivation will follow.

\newcommand{\Core}{\mathsf{Core}}

%TODO: edges are about to fall off, the space is not even
%TODO: I'm not sure using atoms to label flows is a good idea, because, given an atom, one would expect a canonical flow to correspond to it.
%TODO: nobody can understand this definition

\begin{definition}\label{DefFlowCore}
Given an atomic flow
\[
A=\atomicflow
{
(-21, 8.5)*{\afvjm{9}};
(-13, 8)*{\afaidm{}{}{}{}{}{}};
( -5, 8.5)*{\afvjm{9}};
(-18, 0)*{\affr{8}{8}};
(-16, 2)*{a_1};
( -8, 0)*{\affr{8}{8}};
( -6, 2)*{\bar a_1};
( -5,-8.5)*{\afvjm{9}};
(-13,-8)*{\afaium{}{}{}{}{}{}};
(-21,-8.5)*{\afvjm{9}};
%------------
(0,0)*{\cdots};
%------------
(21, 8.5)*{\afvjm{9}};
(13, 8)*{\afaidm{}{}{}{}{}{}};
( 5, 8.5)*{\afvjm{9}};
( 8, 0)*{\affr{8}{8}};
(10, 2)*{a_n};
(18, 0)*{\affr{8}{8}};
(20, 2)*{\bar a_n};
( 5,-8.5)*{\afvjm{9}};
(13,-8)*{\afaium{}{}{}{}{}{}};
(21,-8.5)*{\afvjm{9}};
%------------
(33, 11)*{\afvjm4};
%---
( 33, 5)*{\affr{18}8};
( 28, 5)*{\copy\contrup};
( 33, 5)*{\copy\weakdown};
( 38, 5)*{\copy\contrdown};
( 48, 5)*{\affr88};
( 48, 5)*{\copy\interdown};
%---
( 28, 0)*{\afvjm2};
( 38, 0)*{\afvjm2};
( 48, 0)*{\afvjm2};
%---
( 28,-5)*{\affr88};
( 28,-5)*{\copy\interup};
( 43,-5)*{\affr{18}8};
( 38,-5)*{\copy\contrup};
( 43,-5)*{\copy\weakup};
( 48,-5)*{\copy\contrdown};
%---
( 43,-11)*{\afvjm4};
}\quad,
\]
the \emph{core of $A$} is defined to be
\[
\Core(A)=
\atomicflow
{
(-21,10)*{\afvjm{12}};
(-17,15)*{\afvj2};
(-17,10)*{\affr{6}{8}};
(-17,10)*{\copy\contrup};
(-17, 5)*{\afvjm{2}};
(-18, 0)*{\affr{8}{8}};
(-16, 2)*{a_1};
(-21,-10)*{\afvjm{12}};
(-17,-15)*{\afvj2};
(-17,-10)*{\affr{6}{8}};
(-17,-10)*{\copy\contrdown};
(-17, -5)*{\afvjm{2}};
%
( -9,15)*{\afvj2};
( -9,10)*{\affr{6}{8}};
( -9,10)*{\copy\contrup};
( -9, 5)*{\afvjm{2}};
( -5,10)*{\afvjm{12}};
( -8, 0)*{\affr{8}{8}};
( -6, 2)*{\bar a_1};
( -9,-15)*{\afvj2};
( -9,-10)*{\affr{6}{8}};
( -9,-10)*{\copy\contrdown};
( -9, -5)*{\afvjm{2}};
( -5,-10)*{\afvjm{12}};
( -8, 0)*{\affr{8}{8}};
%------------
(0,0)*{\cdots};
%------------
( 9,15)*{\afvj2};
( 9,10)*{\affr{6}{8}};
( 9,10)*{\copy\contrup};
( 9, 5)*{\afvjm{2}};
( 5,10)*{\afvjm{12}};
( 8, 0)*{\affr{8}{8}};
(10, 2)*{a_n};
( 9,-15)*{\afvj2};
( 9,-10)*{\affr{6}{8}};
( 9,-10)*{\copy\contrdown};
( 9, -5)*{\afvjm{2}};
( 5,-10)*{\afvjm{12}};
( 8, 0)*{\affr{8}{8}};
%
(21,10)*{\afvjm{12}};
(17,15)*{\afvj2};
(17,10)*{\affr{6}{8}};
(17,10)*{\copy\contrup};
(17, 5)*{\afvjm{2}};
(18, 0)*{\affr{8}{8}};
(20, 2)*{\bar a_n};
(21,-10)*{\afvjm{12}};
(17,-15)*{\afvj2};
(17,-10)*{\affr{6}{8}};
(17,-10)*{\copy\contrdown};
(17, -5)*{\afvjm{2}};
%---------
(33, 12.5)*{\afvjm7};
%---
( 33, 5)*{\affr{18}8};
( 28, 5)*{\copy\contrup};
( 33, 5)*{\copy\weakdown};
( 38, 5)*{\copy\contrdown};
( 48, 5)*{\affr88};
( 48, 5)*{\copy\interdown};
%---
( 28, 0)*{\afvjm2};
( 38, 0)*{\afvjm2};
( 48, 0)*{\afvjm2};
%---
( 28,-5)*{\affr88};
( 28,-5)*{\copy\interup};
( 43,-5)*{\affr{18}8};
( 38,-5)*{\copy\contrup};
( 43,-5)*{\copy\weakup};
( 48,-5)*{\copy\contrdown};
%---
( 43,-12.5)*{\afvjm7};
}\quad,
\]
where the subflow of $A$ labelled $a_i$ (resp., $\bar a_i$) is isomorphic to the subflow of $\Core(A)$ labelled $a_i$ (resp., $\bar a_i$) for every $1\leq i\leq n$ and the rightmost subflow of $A$ is isomorphic to the rightmost sublfow of $\Core(A)$.
\end{definition}

We now define the core of a derivation. The core is weakly streamlined at the expense of new atoms in the premiss and conclusion. In the next section we will see how we can glue several cores together to eliminate these atoms, but still obtain a weakly streamlined derivation.

%TODO: at this point the reader is lost, what are you trying to convey?

\begin{definition}\label{DefCore}
Given a derivation $\vlder{\Phi}{}{\beta}{\alpha}$ with associated atomic flow $A$, a \emph{core of\/ $\Phi$} is defined to be a derivation $\vlder{\Core(\Phi)}{}{\vls[\beta.(a_n.{\bar a_n}).\cdots.(a_1.{\bar a_1})]}{\vls([a_1.{\bar a_1}].\cdots.[a_n.{\bar a_n}].\alpha)}$ with associated atomic flow $\Core(A)$.
\end{definition}

To obtain the core of a derivation is straightforward and a detailed proof can be found in the full version of this paper.

\begin{theorem}
Given a derivation $\vlder{\Phi}{}{\beta}{\alpha}$, the derivation $\Core(\Phi)$ exists.
\end{theorem}

\subsection{The Normaliser}

%TODO: Still, obtaining the core requires knowledge of the original derivation, so, it's not quite clear what is the moral point you are trying to make.

We now show how to build a weakly streamlined derivation. We present the \emph{normaliser}, which is a `skeleton' where we can plug copies of the core of the derivation we want to normalise. The novelty of this method is that no knowledge of the shape of the core is required. The only information needed to build the normaliser is the number of atoms which must be eliminated from the premiss and conclusion of the core in order to obtain the desired derivation.

\newcommand{\contr}{\mathsf{c}}
\newcommand{\cod}{{\contr{\downarrow}}}
\newcommand{\cou}{{\contr{\uparrow}}}

%TODO: the normaliser takes a derivation so it does depend on the core, contradict what was just said

\newcommand{\Norm}{\mathsf{Norm}}

\begin{definition}
For every $n\geq 0$ the \emph{normaliser}, $\Norm(\Phi,a_1,\dots,a_n)$, is an operator taking as input a sequence of atoms and a derivation of the form
\[
\vlder{\Phi}{}{\vls[\beta.(a_n.{\bar a_n}).\cdots.(a_1.{\bar a_1})]}{\vls([a_1.{\bar a_1}].\cdots.[a_n.{\bar a_n}].\alpha)}\quad,
\]
where $\alpha$ and $\beta$ are formulae and returning a derivation of the form
\[
\vlder{\Norm(\Phi,a_1,\dots,a_n)}{}{\beta}{\alpha}\quad.
\]

We define $\Norm$ inductively on the number of arguments. Let $\Norm(\Phi)=\Phi$ and for $n>0$ let $\Norm(\Phi,a_1,\dots,a_n)$ be
\newbox\DeltaTopK
\setbox\DeltaTopK=
\hbox{$
\vlderivation
{
 \vlde{\Norm(\Phi,a_1,\dots,a_{n-1})}{}{\vls[\beta.(\vlinf{\awu}{}{\ttt}{a_n}.\bar a_n)]}
 {
  \vlhy{\vls(\vlinf{\aid}{}{\vls[a_n.\bar a_n]}{\ttt}.\alpha)}
 }
}$
}
\newbox\DeltaBotK
\setbox\DeltaBotK=
\hbox{
$\vlderivation
{
 \vlde{\Norm(\Phi,a_1,\dots,a_{n-1})}{}{\vls[\beta.\vlinf{\aiu}{}{\fff}{\vls(a_n.\bar a_n)}]}
 {
  \vlhy{\vls([a_n.\vlinf{\awd}{}{\bar a_n}{\fff}].\alpha)}
 }
}$
}
\newbox\DeltaK
\setbox\DeltaK=
\hbox{$
\vlderivation
{
 \vlde{\Norm(\Phi,a_1,\dots,a_{n-1})}{}{\vls[\beta.(a_n.\vlinf{\awu}{}{\ttt}{\bar a_n})]}
 {
  \vlhy{\vls([\vlinf{\awd}{}{a_n}{\fff}.\bar a_n].\alpha)}
 }
}$
}
\[
\vlderivation
{
 \vlin{}{}{\beta}
 {
  \vlin{\swi}{}{\vls[\vlinf{}{}{\beta}{\vls[\beta.\beta]}.\box\DeltaBotK]}
  {
   \vlin{\swi}{}{\vls([\beta.\box\DeltaK].\alpha)}
   {
    \vlin{}{}{\vls(\box\DeltaTopK.\vlinf{}{}{\vls(\alpha.\alpha)}{\alpha})}
    {
     \vlhy{\alpha}
    }
   }
  }
 }
}\quad.
\]
\end{definition}

\begin{remark}\label{RemFlowNorm}
If the atomic flow of $\vlder{\Norm(\Phi,a_1,\dots,a_{n-1})}{}{\vls[\beta.(a_n.\bar a_n)]}{\vls([a_n.\bar a_n].\alpha)}$ is
\[
\atomicflow
{
(-8, 6)*{\afvjm{4}};
(-2, 6)*{\afvju{4}{a_n}{}};
( 2, 6)*{\afvju{4}{}{\bar a_n}};
( 8, 6)*{\afvjm{4}};
(-5, 0)*{\affr{8}{8}};
(-3, 2)*{A};
( 5, 0)*{\affr{8}{8}};
( 7, 2)*{B};
( 8,-6)*{\afvjm{4}};
(-2,-6)*{\afvjd{4}{a_n}{}};
( 2,-6)*{\afvjd{4}{}{\bar a_n}};
(-8,-6)*{\afvjm{4}};
}\quad,
\]
then the atomic flow of $\vlder{\Norm(\Phi,a_1,\dots,a_n)}{}{\beta}{\alpha}$ is
\[
\atomicflow
{
% cocontractions
%  outer
(-13.5,36.5)*{\afacumexsqcol{}{}{}{}{}{}{33}{4}{}{Green}{Green}};
(  2.5,36.5)*{\afacumexsqcol{}{}{}{}{}{}{33}{4}{}{Green}{Green}};
%  inner
( -8, 13)*{\afvjmcol{18}{Green}};
( 14,  0)*{\afvjmcol{44}{Green}};
(  3, 26)*{\afacumnwexsqcol{}{}{}{}{11}{2}{Green}{Green}};
(  8, 13)*{\afvjm{18}};
( 30, 0)*{\afvjmcol{44}{Green}};
( 19, 26)*{\afacumnwexsqcol{}{}{}{}{11}{2}{}{Green}};
% top boxes
(-22, 34)*{\afaidcol{}{}{}{}{}{}{Red}{Red}};
(-27, 26)*{\affr{8}{8}};
(-25, 28)*{A_1};
(-17, 26)*{\affr{8}{8}};
(-15, 28)*{B_1};
(-24, 18)*{\afawucol{}{}{}{}{}{Red}};
( -9, 13)*{\afcjlcol{22}{18}{Red}};
% middle boxes
( -2,  8)*{\afawdcol{}{}{}{}{}{Green}};
(-5, 0)*{\affr{8}{8}};
(-3, 2)*{A_2};
( 5, 0)*{\affr{8}{8}};
( 7, 2)*{B_2};
(  2, -8)*{\afawucol{}{}{}{}{}{Red}};
% bottom boxes
( 22,-34)*{\afaiucol{}{}{}{}{}{}{Green}{Green}};
( 17,-26)*{\affr{8}{8}};
( 19,-24)*{A_3};
( 27,-26)*{\affr{8}{8}};
( 29,-24)*{B_3};
( 24,-18)*{\afawdcol{}{}{}{}{}{Green}};
(  9,-13)*{\afcjlcol{22}{18}{Green}};
% contractions
%  inner
( -8,-12.75)*{\afvjm{17.5}};
(-30,0.25)*{\afvjmcol{43.5}{Red}};
(-19,-27.5)*{\afacdmnwexsqcol{}{}{}{}{11}{2}{Red}{}};
(  8,-12.75)*{\afvjmcol{17.5}{Red}};		
(-14,0.25)*{\afvjmcol{43.5}{Red}};
( -3,-27.5)*{\afacdmnwexsqcol{}{}{}{}{11}{2}{Red}{Red}};
%  outer
( 13.5,-36)*{\afacdmexsqcol{}{}{}{}{}{}{33}{4}{Red}{}{Red}};
( -2.5,-36)*{\afacdmexsqcol{}{}{}{}{}{}{33}{4}{Red}{}{Red}};
}\quad,
\]
where $A_1$, $A_2$, $A_3$ are isomorphic to $A$ and $B_1$, $B_2$, $B_3$ are isomorphic to $B$ and all the edges that might be in paths from the evidenced interaction vertex are colored in red and all the edges that might be in paths from the evidenced cointeraction vertex are colored in green.
\end{remark}

By studying Definition~\ref{DefFlowCore} and Proposition~\ref{RemFlowNorm} we can see how a weakly streamlined derivation can be built. The proof is given in the full version of this paper.

\begin{theorem}
Given a derivation $\vlder{\Phi}{}{\beta}{\alpha}$, there are atoms $a_1,\dots,a_n$ in $\Phi$ such that $\vlder{\Norm(\Core(\Phi),a_1,\dots,a_n)}{}{\beta}{\alpha}$ is weakly streamlined.
\end{theorem}

%===========================================
\section{Conclusion}

\bibliographystyle{alpha}
\bibliography{di-biblio}

\end{document}