\documentclass[a4paper]{llncs}

\usepackage[lutzsyntax]{virginialake}\aftrianglefalse
%\usepackage[urw-garamond]{mathdesign}


\begin{document}

\title{Normalisation Control in Deep Inference\\ via Atomic Flows II}

\author{Alessio Guglielmi\inst{1,2} \and Tom Gundersen\inst{1}}
\institute{University of Bath, Bath BA2 7AY, UK \and INRIA, 615, rue du Jardin Botanique, 54603 Villers-les-Nancy Cedex, France}

\thanks{This work was in part funded by an Overseas Research Scholarship and a Research Studentship, both from the University of Bath, and by the British Council Alliance Programme.}

\maketitle

%TODO: The introduction should have a better underlying scheme (what do you want to convey here?) and should contain some moral statements about what we do here.

%What?
We are interested in normalising derivations in propositional logic. To us, normalisation is the elimination of `unreachable' parts of derivations. An abstract depiction of an example of an unreachable part of a derivation is the following
\[
\atomicflow{
( 0  , 4  )*{\afaidnw{}{}};
( 6  , 3  )*{\afvjd6{}a};
( 0  , 2.3)*{\aflabelright{\bar a}};
( 2  , 2  )*{\afvj4};
(-2  , 1  )*{\afvju6a{}};
( 4  ,-2  )*{\afaiunw{}{}};
(-3.5, 0  )*{\invisiblemark};
( 7.5, 0  )*{\invisiblemark}}
\quad,
\]
where the atom $\bar a$ is first created by an axiom and then destroyed by a cut (represented by horizontal bars), hence never reaching the premiss nor the conclusion of the derivation.

%TODO: why, which problems, too generic
%TODO: language for what
%Why?
%We believe that understanding normalisation is crucial for solving problems related to complexity and identity of proofs. To argue about these complicated notions a simple language is needed and in particular a language without bureaucracy.

%State of affairs
At the core of our work lies \emph{atomic flows}, which were introduced in \cite{GuglGund:07:Normalis:lr}. Atomic flows are graphs obtained by removing all logical information from derivations and only retaining the structural part. They are largely syntax independent and bureaucracy free. It was shown how atomic flows are useful in defining new normal forms and in arguing about normalisation.

In particular, \emph{streamlining} was defined based on atomic flows. Streamlining is an up-down symmetric generalisation of cut elimination, which applies to derivations as well as to proofs. Intuitively a derivation is streamlined if every path in the associated atomic flow can be extended to reach the top or the bottom of the atomic flow. Seen from the point of view of derivations it means that if you pick an atom occurrence from a streamlined derivation and start tracing it both up and down, you must reach either the premiss or the conclusion. Since a proof has no atoms in its premiss (only the unit `true'), tracing atom occurrences upwards from a cut can not lead to the premiss. Hence,  a streamlined proof is cut free.

In this paper we present a new streamlining procedure based on a global transformation. The moral idea is that the only information needed to guide streamlining is information about the connectedness of axioms and cuts. Furthermore, no transformation of the original derivation is required, except for removing the connected axioms and cuts. Due to the symmetries of deep inference, we are able to paste together copies of the original derivation, with some axioms and cuts removed, to obtain a derivation in normal form.

The novelty of this method is that the complexity of the procedure is decided by the number of atoms which occur in axioms and cuts that are eliminated, as opposed to the number of cuts. Furthermore, given the number of atoms being eliminated we can build a `skeleton' where we can plug the original derivation to obtain a streamlined derivation. This skeleton only depends on the number of atoms and not on the structure of the derivation being streamlined. As was also true of our previous results, logical inference rules and logical connectives do not play a role, which goes contrary to what is traditionally expected from cut elimination.

\newcommand{\SKS}{\mathsf{SKS}}
%Constraints
The results in this paper are presented in the deep inference formalism the calculus of structures \cite{Gugl:06:A-System:kl} and system $\SKS$ \cite{BrunTiu:01:A-Local-:mz}, but we strive at generality and it should not be difficult to adapt our results to any deep inference formalism and any propositional system as long as we have atomic structural rules and linear logical rules (something we always expect in deep inference).

For the full version of this paper please refer to \url{http://www.jklm.no/teg/AFII.pdf}
%===============================================================================


\bibliographystyle{alpha}
\bibliography{di-biblio}

\end{document}