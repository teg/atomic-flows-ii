\documentclass[a4paper]{amsart}

%                  Trash .aux file after toggling
\usepackage{stmaryrd}
\usepackage{graphicx}
\usepackage[margin=1in]{geometry}
\usepackage[lutzsyntax]{virginialake}\aftrianglefalse
\usepackage[pdfborder={0 0 0}]{hyperref}
\usepackage[urw-garamond]{mathdesign}

%--------- Theorem etc
\newtheorem{thm}{Theorem}[section]
\newtheorem{cor}[thm]{Corollary}
\newtheorem{lem}[thm]{Lemma}
\newtheorem{pro}[thm]{Proposition}

\theoremstyle{remark}
\newtheorem{rem}[thm]{Remark}
\newtheorem{exa}[thm]{Example}

\theoremstyle{definition}
\newtheorem{defi}[thm]{Definition}
%---------

\begin{document}

\title[Normalisation Control in Deep Inference   via Atomic Flows II]
      {Normalisation Control in Deep Inference\\ via Atomic Flows II}

\author{Alessio Guglielmi and Tom Gundersen}
%\address{University of Bath, Bath BA2 7AY, UK}

\thanks{This work was in part funded by an Overseas Research Scholarship and a Research Studentship, both from the University of Bath, and by the British Council Alliance Programme.}

\keywords{Normalisation, deep inference, cut elimination, atomic flows}

\subjclass{F.4.1 Mathematical Logic---Proof theory}

% \begin{abstract}
% \end{abstract}

\maketitle

%===============================================================================
\section{Introduction}


%===============================================================================
\section{Background on Deep Inference}

\newcommand{\fff}{\mathsf f}
\newcommand{\ttt}{\mathsf t}
\newcommand{\ot}{\mathbin\shortleftarrow}

% TODO: check that we only every use the variable names in the following definition

%---------------------------------------
\begin{defi}\label{DefFormulae}
\emph{Formulae}, $\alpha$, $\beta$, $\gamma$, $\delta$ are freely built from: \emph{units}, $\fff$ (false), $\ttt$ (true); \emph{atoms}, $a$, $b$, $c$, $d$; \emph{disjunction} and \emph{conjunction}, ${\vlsbr[\alpha.\beta]}$ and $\vlsbr(\alpha.\beta)$. On the set of atoms a (non-identical) involution $\bar\cdot$ is defined, and dual atom occurrences, as $a$ and $\bar a$, can appear in formulae. We denote \emph{contexts}, \emph{i.e.}, formulae with a hole, by $\xi\vlhole$ and $\zeta\vlhole$; we also use \emph{multiple} contexts, $\xi\vlhole\cdots\vlhole$, \emph{i.e.}, formulae with many holes; for example, if $\xi\{a\}$ is $\vls(b.[a.c])$, then $\xi\vlhole$ is $\vls(b.[\vlhole.c])$, $\xi\{b\}$ is $\vls(b.[b.c])$ and $\xi\vlscn(a.d)$ is $\vls(b.[(a.d).c])$; if $\xi\{a\}\{b\}\{c\}$ is $\vls(b.[(a.d).c])$ then $\xi\{b\}\{c\}\{a\}$ is $\vls(c.[(b.d).a])$.
\end{defi}

%---------------------------------------
\begin{rem}
Negation is only defined for atoms, which is not a limitation thanks to De Morgan laws.
\end{rem}

\newcommand{\one}{{\mathchoice{\scriptstyle\mathbf1}
                              {\scriptstyle\mathbf1}
                              {\scriptstyle\mathbf1}
                              {\scriptscriptstyle\mathbf1}}}
\newcommand{\two}{{\mathchoice{\scriptstyle\mathbf2}
                              {\scriptstyle\mathbf2}
                              {\scriptstyle\mathbf2}
                              {\scriptscriptstyle\mathbf2}}}
%---------------------------------------
\begin{defi}\label{DefDerivation}
\emph{Inference rules}, $\rho$, have one \emph{premiss} and one \emph{conclusion}, and their \emph{instances} are used in \emph{inference steps} to rewrite inside formulae. A \emph{derivation}, $\Phi$, from $\alpha$ (\emph{premiss}) to $\beta$ (\emph{conclusion}) is a chain of inference steps with $\alpha$ at the top and $\beta$ at the bottom, and is usually indicated by $\vlder{\Phi}{\mathcal S}{\beta}{\alpha}$, where $\mathcal S$ is the name of the deductive system or a set of inference rules; a \emph{proof} is a derivation from $\ttt$; besides $\Phi$, we denote derivations with $\Psi$. We denote with $\xi\{\Phi\}$ the result of including every formula of the derivation $\vlder{\Phi}{}{\beta}{\alpha}$ into the context $\xi\vlhole$. Since we adopt deep inference, $\vlder{\xi\{\Phi\}}{}{\xi\{\beta\}}{\xi\{\alpha\}}$ is a valid derivation. Furthermore, $\xi\left\{\vlder{}{}{\beta_1}{\alpha_1}\right\}\cdots\left\{\vlder{}{}{\beta_n}{\alpha_n}\right\}$ denotes
\[
\vlderivation
{
 \vlin{=}{}{\xi\{\beta_1\}\cdots\{\beta_{n-1}\}\left\{\vlder{}{}{\beta_n}{\alpha_n}\right\}}
 {
  \vlin{=}{}{\vdots}
  {
   \vlin{=}{}{\xi\{\beta_1\}\cdots\{\beta_{i-1}\}\left\{\vlder{}{}{\beta_i}{\alpha_i}\right\}\{\alpha_{i+1}\}\cdots\{\alpha_n\}}
   {
    \vlin{=}{}{\vdots}
    {
     \vlhy{\xi\left\{\vlder{}{}{\beta_1}{\alpha_1}\right\}\{\alpha_2\}\cdots\{\alpha_n\}}
    }
   }
  }
 }
}\quad.
\]
\end{defi}

\newcommand{\KS}{\mathsf{KS}}
\newcommand{\SKS}{\mathsf{SKS}}

\newcommand{\ai}{\mathsf{ai}}
\newcommand{\aw}{\mathsf{aw}}
\newcommand{\ac}{\mathsf{ac}}
\newcommand{\aid}{{\ai{\downarrow}}}
\newcommand{\awd}{{\aw{\downarrow}}}
\newcommand{\acd}{{\ac{\downarrow}}}
\newcommand{\aiu}{{\ai{\uparrow}}}
\newcommand{\awu}{{\aw{\uparrow}}}
\newcommand{\acu}{{\ac{\uparrow}}}
\newcommand{\swi}{\mathsf{s}}
\newcommand{\med}{\mathsf{m}}
%---------------------------------------
\begin{defi}
System $\SKS$ in the calculus of structures is defined by the following \emph{structural} rules:
\[
\begin{array}{@{}c@{}c@{}c@{}}
      \vlinf{\aid}{}{\vls[a.{\bar a}]}{\ttt}&
\qquad\vlinf{\awd}{}a\fff&
\qquad\vlinf{\acd}{}a{\vls[a.a]}\\
\noalign{\smallskip}
      \emph{interaction}&
\qquad\emph{weakening}&
\qquad\emph{contraction}\\
\noalign{\bigskip}
      \vlinf{\aiu}{}\fff{\vls(a.{\bar a})}&
\qquad\vlinf{\awu}{}\ttt a&
\qquad\vlinf{\acu}{}{\vls (a.a)}a\\
\noalign{\smallskip}
      \emph{cointeraction}&
\qquad\emph{coweakening}&
\qquad\emph{cocontraction}\\
\end{array}\quad,
\]
and by the two \emph{logical} rules:
\[
\begin{array}{@{}c@{}c@{}}
\vlinf{\swi}{}{\vls[(\alpha.\beta).\gamma]}{\vls(\alpha.[\beta.\gamma])}&\qquad
\vlinf{\med}{}{\vls([\alpha.\gamma].[\beta.\delta])}
              {\vls[(\alpha.\beta).(\gamma.\delta)]}\\
\noalign{\smallskip}
\emph{switch}&\qquad\emph{medial}\\
\end{array}\quad.
\]
The rule cointeraction is also called an (\emph{atomic}) \emph{cut}. In addition to the rules shown, there is a rule $\vldownsmash{\vlinf={}\delta\gamma}$, such that $\gamma$ and $\delta$ are opposite sides in one of the following equations:
\vlstore{
\vls[\alpha.\beta]         &=\vls[\beta.\alpha]         \quad,&
\vls[\alpha.\fff]          &=\vls[\alpha]               \quad,\\
\vls(\alpha.\beta)         &=\vls(\beta.\alpha)         \quad,&
\vls(\alpha.\ttt)          &=\vls(\alpha)               \quad,\\
\vls[[\alpha.\beta].\gamma]&=\vls[\alpha.[\beta.\gamma]]\quad,&
\vls[\ttt.\ttt]            &=\vls[\ttt]                 \quad,\\
\vls((\alpha.\beta).\gamma)&=\vls(\alpha.(\beta.\gamma))\quad,&
\vls(\fff.\fff)            &=\vls(\fff)                 \quad\vldot}
\begin{align*}
\vlread
\end{align*}
We do not always show the instances of rule $=$, and when we do show them, we gather several contiguous instances into one. System $\KS$ is the same as $\SKS$, but without the rules $\aiu$, $\awu$ and $\acu$. A \emph{cut-free} derivation is a derivation where $\aiu$ is not used. All derivations in this paper are in $\SKS$, unless indicated otherwise.
\end{defi}

\begin{rem}
The representations of $\SKS$ derivations in this paper are sometimes ambiguous. This is not a problem, as the derivations themselves are well defined and we can always get a more detailed representation should we need to, for instance when extracting atomic flows from derivations. The ambiguity arises because we omit or collapse equations and because we do not distinguish the different equation rules. In particular we need to know if
\[
\vlinf{=}{}{\vls((a.a).a)}{\vls(a.(a.a))}\quad,
\]
is an application of associativity $\left(\vlinf{=}{}{\vls((a.b).c)}{\vls(a.(b.c))}\right)$ or commutativity $\left(\vlinf{=}{}{\vls((b.c).a)}{\vls(a.(b.c))}\right)$ to be able to map atom occurrences in the premiss to atom occurrences in the conclusion.
\end{rem}

% TODO: find the right place for this example

\begin{exa}\label{ExaFormalismA}
Consider the $\SKS$ derivation
\[
\vls[\vlinf{\acd}{}{a}{\vls[a.a]}.\vlinf{\aiu}{}{\fff}{\vls(\vlinf{\acd}{}{b}{\vls[b.b]}.\bar b)}]\quad,
\]
which is a shorthand for
\[
\vlderivation
{
 \vlin{\aiu}{}{a}
 {
  \vlin{\acd}{}{\vls[a.(b.\bar b)]}
  {
   \vlin{\acd}{}{\vls[a.([b.b].\bar b)]}
   {
    \vlhy{\vls[a.a.([b.b].\bar b)]}
   }
  }
 }
}.
\]
It is implicit in Definition~\ref{DefFormulae} that the ordering of holes in a formula context with multiple holes is given. This decides the order in which we `sequentialise' the short-hand representation of a derivation in Definition~\ref{DefDerivation}. In this example we used a left-to-right ordering, but we could just as well have used a right-to-left ordering:
\[
\vlderivation
{
 \vlin{\acd}{}{a}
 {
  \vlin{\aiu}{}{\vls[a.a]}
  {
   \vlin{\acd}{}{\vls[a.a.(b.\bar b)]}
   {
    \vlhy{\vls[a.a.([b.b].\bar b)]}
   }
  }
 }
}.
\]
In Formalism A, a new deep-inference formalism currently under development, the above shorthand is a representative of the equivalence class containing both the left-to-right and the right-to-left orderings, in addition to the intermediate:
\[
\vlderivation
{
 \vlin{\aiu}{}{a}
 {
  \vlin{\acd}{}{\vls[a.(b.\bar b)]}
  {
   \vlin{\acd}{}{\vls[a.a.(b.\bar b)]}
   {
    \vlhy{\vls[a.a.([b.b].\bar b)]}
   }
  }
 }
}.
\]
For the purposes of this paper, which of these three derivations we use is inessential, as the atomic flows associated with each of them are are the same. For simplicity we let the shorthand above refer to one unique derivation, but we could just as well have worked in Formalism A.
\end{exa}

%===============================================================================

\section{Background on Atomic Flows}

\newbox\contrup\setbox\contrup=\hbox{$
   \divide\atflowunit by5\multiply\atflowunit by3\afsetunits
   \atomicflow{(0,0)*{\afacu{}{}{}{}{}{}}}$}
\newbox\contrdown\setbox\contrdown=\hbox{$
   \divide\atflowunit by5\multiply\atflowunit by3\afsetunits
   \atomicflow{(0,0)*{\afacd{}{}{}{}{}{}}}$}
\newbox\interdown\setbox\interdown=\hbox{$
   \divide\atflowunit by5\multiply\atflowunit by3\afsetunits
   \atomicflow{(0,0)*{\afaid{}{}{}{}{}{}}}$}
\newbox\interup\setbox\interup=\hbox{$
   \divide\atflowunit by5\multiply\atflowunit by3\afsetunits
   \atomicflow{(0,0)*{\afaiu{}{}{}{}{}{}}}$}
\newbox\weakdown\setbox\weakdown=\hbox{$
   \divide\atflowunit by5\multiply\atflowunit by3\afsetunits
   \atomicflow{(0,0)*{\afawd{}{}{}{}{}{}}}$}
\newbox\weakup\setbox\weakup=\hbox{$
   \divide\atflowunit by5\multiply\atflowunit by3\afsetunits
   \atomicflow{(0,0)*{\afawu{}{}{}{}{}{}}}$}
%===============================================================================
\section{Strongly Normalising Streamlining}

%---------------------------------------
\begin{defi}
An $\SKS$ derivation is \emph{streamlined} if, in its associated atomic flow, there are no paths from interaction or weakening vertices to cointeraction or coweakening vertices.
\end{defi}

%---------------------------------------
\begin{rem}\label{RemStr}
It immediately follows from the definition that the diagram below describes the shape of a streamlined derivation:
\[
\atomicflow{
(-10,11)*{\afvjm4};
%---
(-15, 5)*{\copy\contrup};
(-10, 5)*{\affr{28}8};
( -5, 5)*{\copy\contrdown};
( 10, 5)*{\copy\interdown};
( 10, 5)*{\affr88};
( 20, 5)*{\copy\weakdown};
( 20, 5)*{\affr88};
%---
(-20, 0)*{\afvjm2};
(-10, 0)*{\afvjm2};
(  0, 0)*{\afvjm2};
( 10, 0)*{\afvjm2};
( 20, 0)*{\afvjm2};
%---
(-20,-5)*{\copy\weakup};
(-20,-5)*{\affr88};
(-10,-5)*{\copy\interup};
(-10,-5)*{\affr88};
(  5,-5)*{\copy\contrup};
( 10,-5)*{\affr{28}8};
( 15,-5)*{\copy\contrdown};
%---
(  10,-11)*{\afvjm4};
}\quad.
\]
\end{rem}

%---------------------------------------
\begin{defi}
An $\SKS$ derivation is \emph{weakly streamlined} if, in its associated atomic flow, there are no paths from interaction vertices to cointeraction vertices.
\end{defi}

%---------------------------------------
\begin{rem}\label{RemStr}
It immediately follows from the definition that the diagram below describes the shape of a weakly streamlined derivation:
\[
\atomicflow{
(-5, 11)*{\afvjm4};
%---
( -5, 5)*{\affr{18}8};
(-10, 5)*{\copy\contrup};
( -5, 5)*{\copy\weakdown};
(  0, 5)*{\copy\contrdown};
( 10, 5)*{\affr88};
( 10, 5)*{\copy\interdown};
%---
(-10, 0)*{\afvjm2};
(  0, 0)*{\afvjm2};
( 10, 0)*{\afvjm2};
%---
(-10,-5)*{\affr88};
(-10,-5)*{\copy\interup};
(  5,-5)*{\affr{18}8};
(  0,-5)*{\copy\contrup};
(  5,-5)*{\copy\weakup};
( 10,-5)*{\copy\contrdown};
%---
(  5,-11)*{\afvjm4};
}\quad.
\]
\end{rem}

\subsection{The Core}

\newcommand{\Core}{\mathsf{Core}}

\begin{defi}\label{DefFlowCore}
Given an atomic flow
\[
A=\atomicflow
{
(-21, 8.5)*{\afvjm{9}};
(-13, 8)*{\afaidm{}{}{}{}{}{}};
( -5, 8.5)*{\afvjm{9}};
(-18, 0)*{\affr{8}{8}};
(-16, 2)*{a_1};
( -8, 0)*{\affr{8}{8}};
( -6, 2)*{\bar a_1};
( -5,-8.5)*{\afvjm{9}};
(-13,-8)*{\afaium{}{}{}{}{}{}};
(-21,-8.5)*{\afvjm{9}};
%------------
(0,0)*{\cdots};
%------------
(21, 8.5)*{\afvjm{9}};
(13, 8)*{\afaidm{}{}{}{}{}{}};
( 5, 8.5)*{\afvjm{9}};
( 8, 0)*{\affr{8}{8}};
(10, 2)*{a_n};
(18, 0)*{\affr{8}{8}};
(20, 2)*{\bar a_n};
( 5,-8.5)*{\afvjm{9}};
(13,-8)*{\afaium{}{}{}{}{}{}};
(21,-8.5)*{\afvjm{9}};
%------------
(33, 11)*{\afvjm4};
%---
( 33, 5)*{\affr{18}8};
( 28, 5)*{\copy\contrup};
( 33, 5)*{\copy\weakdown};
( 38, 5)*{\copy\contrdown};
( 48, 5)*{\affr88};
( 48, 5)*{\copy\interdown};
%---
( 28, 0)*{\afvjm2};
( 38, 0)*{\afvjm2};
( 48, 0)*{\afvjm2};
%---
( 28,-5)*{\affr88};
( 28,-5)*{\copy\interup};
( 43,-5)*{\affr{18}8};
( 38,-5)*{\copy\contrup};
( 43,-5)*{\copy\weakup};
( 48,-5)*{\copy\contrdown};
%---
( 43,-11)*{\afvjm4};
}\quad,
\]
the \emph{core of $A$} is defined to be
\[
\Core(A)=
\atomicflow
{
(-21,10)*{\afvjm{12}};
(-17,15)*{\afvj2};
(-17,10)*{\affr{6}{8}};
(-17,10)*{\copy\contrup};
(-17, 5)*{\afvjm{2}};
(-18, 0)*{\affr{8}{8}};
(-16, 2)*{a_1};
(-21,-10)*{\afvjm{12}};
(-17,-15)*{\afvj2};
(-17,-10)*{\affr{6}{8}};
(-17,-10)*{\copy\contrdown};
(-17, -5)*{\afvjm{2}};
%
( -9,15)*{\afvj2};
( -9,10)*{\affr{6}{8}};
( -9,10)*{\copy\contrup};
( -9, 5)*{\afvjm{2}};
( -5,10)*{\afvjm{12}};
( -8, 0)*{\affr{8}{8}};
( -6, 2)*{\bar a_1};
( -9,-15)*{\afvj2};
( -9,-10)*{\affr{6}{8}};
( -9,-10)*{\copy\contrdown};
( -9, -5)*{\afvjm{2}};
( -5,-10)*{\afvjm{12}};
( -8, 0)*{\affr{8}{8}};
%------------
(0,0)*{\cdots};
%------------
( 9,15)*{\afvj2};
( 9,10)*{\affr{6}{8}};
( 9,10)*{\copy\contrup};
( 9, 5)*{\afvjm{2}};
( 5,10)*{\afvjm{12}};
( 8, 0)*{\affr{8}{8}};
(10, 2)*{a_n};
( 9,-15)*{\afvj2};
( 9,-10)*{\affr{6}{8}};
( 9,-10)*{\copy\contrdown};
( 9, -5)*{\afvjm{2}};
( 5,-10)*{\afvjm{12}};
( 8, 0)*{\affr{8}{8}};
%
(21,10)*{\afvjm{12}};
(17,15)*{\afvj2};
(17,10)*{\affr{6}{8}};
(17,10)*{\copy\contrup};
(17, 5)*{\afvjm{2}};
(18, 0)*{\affr{8}{8}};
(20, 2)*{\bar a_n};
(21,-10)*{\afvjm{12}};
(17,-15)*{\afvj2};
(17,-10)*{\affr{6}{8}};
(17,-10)*{\copy\contrdown};
(17, -5)*{\afvjm{2}};
%---------
(33, 12.5)*{\afvjm7};
%---
( 33, 5)*{\affr{18}8};
( 28, 5)*{\copy\contrup};
( 33, 5)*{\copy\weakdown};
( 38, 5)*{\copy\contrdown};
( 48, 5)*{\affr88};
( 48, 5)*{\copy\interdown};
%---
( 28, 0)*{\afvjm2};
( 38, 0)*{\afvjm2};
( 48, 0)*{\afvjm2};
%---
( 28,-5)*{\affr88};
( 28,-5)*{\copy\interup};
( 43,-5)*{\affr{18}8};
( 38,-5)*{\copy\contrup};
( 43,-5)*{\copy\weakup};
( 48,-5)*{\copy\contrdown};
%---
( 43,-12.5)*{\afvjm7};
}\quad,
\]
where the subflow of $A$ labelled $a_i$ (resp., $\bar a_i$) is isomorphic to the sublfow of $\Core(A)$ labelled $a_i$ (resp., $\bar a_i$) for every $1\leq i\leq n$ and the rightmost subflow of $A$ is isomorphic to the rightmost sublfow of $\Core(A)$.
\end{defi}

\begin{defi}\label{DefCore}
Given a derivation $\vlder{\Phi}{}{\beta}{\alpha}$ with associated atomic flow $A$, a \emph{core of\/ $\Phi$} is defined to be a derivation $\vlder{\Core(\Phi)}{}{\vls[\beta.(a_n.{\bar a_n}).\cdots.(a_1.{\bar a_1})]}{\vls([a_1.{\bar a_1}].\cdots.[a_n.{\bar a_n}].\alpha)}$ with associated atomic flow $\Core(A)$.
\end{defi}

\begin{pro}\label{PropStreamlinedCore}
Any $\Core(\Phi)$ is weakly streamlined.
\end{pro}

\begin{lem}\label{LemSuperSwitch}
Given a context $\xi\vlhole$ and a formula $\alpha$ there exist derivations $\vlder{}{\{\swi\}}{\xi\{\alpha\}}{\vls(\alpha.\xi\{\ttt\})}$ and $\vlder{}{\{\swi\}}{\vls[\xi\{\fff\}.\alpha]}{\xi\{\alpha\}}$.
\end{lem}

\begin{proof}
We show how to construct the first derivation, the second one can be done by symmetry. We argue by induction on the number of atoms in $\xi\vlhole$. The base case, $\xi\vlhole=\vlhole$, is trivial and the inductive cases are:

\[
\vlderivation
{
 \vlin{=}{}{\xi\{\alpha\}}
 {
  \vlin{\swi}{}{\vls[\vlder{\Psi}{\{\swi\}}{\xi'\{\alpha\}}{\vls(\alpha.\xi'\{\ttt\})}.\beta]}
  {
   \vlin{=}{}{\vls(\alpha.[\xi'\{\ttt\}.\beta])}
   {
    \vlhy{\vls(\alpha.\xi\{\ttt\})}
   }
  }
 }
}\qquad\mbox{and}\qquad
\vlderivation
{
 \vlin{=}{}{\xi\{\alpha\}}
 {
  \vlin{=}{}{\vls(\vlder{\Psi'}{\{\swi\}}{\xi'\{\alpha\}}{\vls(\alpha.\xi'\{\ttt\})}.\beta)}
  {
   \vlhy{\vls(\alpha.\xi\{\ttt\})}
  }
 }
}\quad,
\]
for some $\xi'\vlhole$ and $\beta$ where $\beta$ is not a unit and $\Psi$ and $\Psi'$ exist by the inductive hypothesis.
\end{proof}

\begin{lem}\label{LemDecompInt}
Given a derivation $\vlder{}{}{\beta}{\alpha}$ with associated atomic flow $A$, there exists a derivation
\[
\vlder{}{\SKS\setminus\{\aid,\aiu\}}{\vls[\beta.\vlinf{\aiu}{}{\fff}{\vls(b_m.\bar b_m)}.\cdots.\vlinf{\aiu}{}{\fff}{\vls(b_1.\bar b_1)}]}{\vls(\vlinf{\aid}{}{\vls[a_1.\bar a_1]}{\ttt}.\cdots.\vlinf{\aid}{}{\vls[a_n.\bar a_n]}{\ttt}.\alpha)}
\]
with associated atomic flow $A$, for some atoms $a_1,\dots,a_n,b_1,\dots,b_m$.
\end{lem}

\begin{proof}
Using Lemma~\ref{LemSuperSwitch} apply the following transformations to each of the (co)interaction instances in $\Phi$:
\[
\vlderivation
{
 \vlde{\Psi'}{}{\delta}
 {
  \vlin{\aid}{}{\xi\vlsbr[a.{\bar a}]}
  {
   \vlde{\Psi}{}{\xi\{\ttt\}}
   {
    \vlhy{\gamma}
   }
  }
 }
}\quad\rightarrow\quad
\vlderivation
{
 \vlde{\Psi'}{}{\delta}
 {
  \vlde{}{\{\swi\}}{\xi\vlsbr[a.{\bar a}]}
  {
   \vlhy{\vlsbr(\vlinf{\aid}{}{\vls[a.{\bar a}]}{\ttt}.\vlder{\Psi}{}{\xi\{\ttt\}}{\gamma})}
  }
 }
}\qquad\mbox{and}\qquad
\vlderivation
{
 \vlde{\Psi'}{}{\delta}
 {
  \vlin{\aid}{}{\xi\{\fff\}}
  {
   \vlde{\Psi}{}{\xi\vlsbr(a.{\bar a})}
   {
    \vlhy{\gamma}
   }
  }
 }
}\quad\rightarrow\quad
\vlderivation
{
 \vlde{}{\{\swi\}}{\vlsbr[\vlder{\Psi'}{}{\delta}{\xi\{\fff\}}.\vlinf{\aiu}{}{\fff}{\vls(a.{\bar a})}]}
 {
  \vlde{\Psi}{}{\xi\vlsbr(a.{\bar a})}
  {
   \vlhy{\gamma}
  }
 }
}\quad.
\]
\end{proof}

\begin{lem}\label{LemGenericContraction}
Given a formula $\alpha$ and a positive integer $n$, there exist derivations $\vlder{}{\{\acd,\med\}}{\alpha}{\bigvee_{i=1}^{n}\alpha}$ and $\vlder{}{\{\acu,\med\}}{\bigwedge_{i=1}^{n}\alpha}{\alpha}$.
\end{lem}

\begin{thm}
Given any derivation $\vlder{\Phi}{}{\beta}{\alpha}$, $\Core(\Phi)$ exists.
\end{thm}

\begin{proof}
We build $\Core(\Phi)$ as follows:
\[
\vlderivation
{
 \vlde{\Psi_2}{\{\acd,\med\}}{\vls[\beta.(a_n.{\bar a_n}).\cdots.(a_1.{\bar a_1})]}
 {
  \vlde{\Phi'}{\SKS\setminus\{\aid,\aiu\}}{{\vls[\beta.\vlinf{\aiu}{}{\fff}{\vls(c_l.\bar c_l)}.\cdots.\vlinf{\aiu}{}{\fff}{\vls(c_1.\bar c_1)}.(a_n.\bar a_n).\cdots.(a_n.\bar a_n).\cdots.(a_1.\bar a_1).\cdots.(a_1.\bar a_1)]}}
  {
   \vlde{\Psi_1}{\{\acu,\med\}}{{\vls([a_1.\bar a_1].\cdots.[a_1.\bar a_1].\cdots.[a_n.\bar a_n].\cdots.[a_n.\bar a_n].\vlinf{\aid}{}{\vls[b_1.\bar b_1]}{\ttt}.\cdots.\vlinf{\aid}{}{\vls[b_k.\bar b_k]}{\ttt}.\alpha)}}
   {
    \vlhy{\vls([a_1.{\bar a_1}].\cdots.[a_n.{\bar a_n}].\alpha)}
   }
  }
 }
}\quad,
\]
where $a_1,\dots,a_n$ are distinct and pairwise non-dual atoms and there are no atoms in common between $b_1,\dots,b_k$ and their duals and $c_1,\dots,c_l$ and their duals, $\Phi'$ exists by Lemma~\ref{LemDecompInt} and $\Psi_1$ and $\Psi_2$ exist by Lemma~\ref{LemGenericContraction}.
\end{proof}


\subsection{The Normaliser}


\newcommand{\contr}{\mathsf{c}}
\newcommand{\cod}{{\contr{\downarrow}}}
\newcommand{\cou}{{\contr{\uparrow}}}

\begin{rem}
In the non-atomic version of system $\SKS$ the derivations shown in Lemma~\ref{LemGenericContraction} correspond to repeated applications of (co)contractions. For this reason we sometimes write the inference rules $\vlinf{\cod}{}{\alpha}{\vls[\alpha.\alpha]}$ and $\vlinf{\cou}{}{\vls(\alpha.\alpha)}{\alpha}$ instead of the derivations $\vlder{}{\{\acd,\med\}}{\alpha}{\vls[\alpha.\alpha]}$ and $\vlder{}{\{\acu,\med\}}{\vls(\alpha.\alpha)}{\alpha}$.
\end{rem}

\newcommand{\Norm}{\mathsf{Norm}}

\begin{defi}
The \emph{normaliser}, $\Norm(\Phi,a_1,\dots,a_n)$, is an operator taking as input a sequence of atoms and a derivation of the form
\[
\vlder{\Phi}{}{\vls[\beta.(a_n.{\bar a_n}).\cdots.(a_1.{\bar a_1})]}{\vls([a_1.{\bar a_1}].\cdots.[a_n.{\bar a_n}].\alpha)}\quad,
\]
where $\alpha$ and $\beta$ are formulae and returning a derivation of the form
\[
\vlder{\Norm(\Phi,a_1,\dots,a_n)}{}{\beta}{\alpha}\quad.
\]

We define $\Norm$ inductively on the number of arguments. Let $\Norm(\Phi)=\Phi$ and for $n>0$ let $\Norm(\Phi,a_1,\dots,a_n)$ be
\newbox\DeltaTopK
\setbox\DeltaTopK=
\hbox{$
\vlderivation
{
 \vlde{\Norm(\Phi,a_1,\dots,a_{n-1})}{}{\vls[\beta.(\vlinf{\awu}{}{\ttt}{a_n}.\bar a_n)]}
 {
  \vlhy{\vls(\vlinf{\aid}{}{\vls[a_n.\bar a_n]}{\ttt}.\alpha)}
 }
}$
}
\newbox\DeltaBotK
\setbox\DeltaBotK=
\hbox{
$\vlderivation
{
 \vlde{\Norm(\Phi,a_1,\dots,a_{n-1})}{}{\vls[\beta.\vlinf{\aiu}{}{\fff}{\vls(a_n.\bar a_n)}]}
 {
  \vlhy{\vls([a_n.\vlinf{\awd}{}{\bar a_n}{\fff}].\alpha)}
 }
}$
}
\newbox\DeltaK
\setbox\DeltaK=
\hbox{$
\vlderivation
{
 \vlde{\Norm(\Phi,a_1,\dots,a_{n-1})}{}{\vls[\beta.(a_n.\vlinf{\awu}{}{\ttt}{\bar a_n})]}
 {
  \vlhy{\vls([\vlinf{\awd}{}{a_n}{\fff}.\bar a_n].\alpha)}
 }
}$
}
\[
\vlderivation
{
 \vlin{\cod}{}{\beta}
 {
  \vlin{\swi}{}{\vls[\vlinf{\cod}{}{\beta}{\vls[\beta.\beta]}.\box\DeltaBotK]}
  {
   \vlin{\swi}{}{\vls([\beta.\box\DeltaK].\alpha)}
   {
    \vlin{\cou}{}{\vls(\box\DeltaTopK.\vlinf{\cou}{}{\vls(\alpha.\alpha)}{\alpha})}
    {
     \vlhy{\alpha}
    }
   }
  }
 }
}\quad.
\]
\end{defi}

\begin{thm}
Given a derivation $\Phi$ from $\alpha$ to $\beta$, where the distinct and non-dual atoms $a_1,\dots,a_n$ and their duals are all the atoms that appear in both interaction and cointeraction instances then $\Norm(\Core(\Phi),a_1,\dots,a_n)$ is weakly streamlined.
\end{thm}

% TODO: check for style

\begin{proof}
We prove the slightly stronger claim: ``$\Norm(\Core(\Phi),a_1,\dots,a_k)$ for $0\leq k \leq n$, is weakly streamlined and $a_i$ does not occur in (co)interaction instances for $k< i \leq n$'', by induction on $k$. The base case, $\Norm(\Core(\Phi))=\Core(\Phi)$, follows by Proposition~\ref{PropStreamlinedCore}.

Consider the atomic flow associated with $\Norm(\Core(\Phi),a_1,\dots,a_{k-1})$ where the edges mapped to by $a_k$, $\bar a_k$ instances in the premiss and conclusion, but not in $\alpha$ or $\beta$ are singled out. Since, by the inductive hypothesis, $a_k$ does not occur in (co)interaction instances the atomic flow can be represented as follows:

\[
\atomicflow
{
(-8, 6)*{\afvjm{4}};
(-2, 6)*{\afvju{4}{a_k}{}};
( 2, 6)*{\afvju{4}{}{\bar a_k}};
( 8, 6)*{\afvjm{4}};
(-5, 0)*{\affr{8}{8}};
(-3, 2)*{A};
( 5, 0)*{\affr{8}{8}};
( 7, 2)*{B};
( 8,-6)*{\afvjm{4}};
(-2,-6)*{\afvjd{4}{a_k}{}};
( 2,-6)*{\afvjd{4}{}{\bar a_k}};
(-8,-6)*{\afvjm{4}};
}\quad.
\]

Now consider the atomic flow associated with $\Norm(\Core(\Phi),a_1,\dots,a_k)$,
\[
\atomicflow
{
% cocontractions
%  outer
(-13.5,36.5)*{\afacumexsqcol{}{}{}{}{}{}{33}{4}{}{Green}{Green}};
(  2.5,36.5)*{\afacumexsqcol{}{}{}{}{}{}{33}{4}{}{Green}{Green}};
%  inner
( -8, 13)*{\afvjmcol{18}{Green}};
( 14,  0)*{\afvjmcol{44}{Green}};
(  3, 26)*{\afacumnwexsqcol{}{}{}{}{11}{2}{Green}{Green}};
(  8, 13)*{\afvjm{18}};
( 30, 0)*{\afvjmcol{44}{Green}};
( 19, 26)*{\afacumnwexsqcol{}{}{}{}{11}{2}{}{Green}};
% top boxes
(-22, 34)*{\afaidcol{}{}{}{}{}{}{Red}{Red}};
(-27, 26)*{\affr{8}{8}};
(-25, 28)*{A_1};
(-17, 26)*{\affr{8}{8}};
(-15, 28)*{B_1};
(-24, 18)*{\afawucol{}{}{}{}{}{Red}};
( -9, 13)*{\afcjlcol{22}{18}{Red}};
% middle boxes
( -2,  8)*{\afawdcol{}{}{}{}{}{Green}};
(-5, 0)*{\affr{8}{8}};
(-3, 2)*{A_2};
( 5, 0)*{\affr{8}{8}};
( 7, 2)*{B_2};
(  2, -8)*{\afawucol{}{}{}{}{}{Red}};
% bottom boxes
( 22,-34)*{\afaiucol{}{}{}{}{}{}{Green}{Green}};
( 17,-26)*{\affr{8}{8}};
( 19,-24)*{A_3};
( 27,-26)*{\affr{8}{8}};
( 29,-24)*{B_3};
( 24,-18)*{\afawdcol{}{}{}{}{}{Green}};
(  9,-13)*{\afcjlcol{22}{18}{Green}};
% contractions
%  inner
( -8,-12.75)*{\afvjm{17.5}};
(-30,0.25)*{\afvjmcol{43.5}{Red}};
(-19,-27.5)*{\afacdmnwexsqcol{}{}{}{}{11}{2}{Red}{}};
(  8,-12.75)*{\afvjmcol{17.5}{Red}};
(-14,0.25)*{\afvjmcol{43.5}{Red}};
( -3,-27.5)*{\afacdmnwexsqcol{}{}{}{}{11}{2}{Red}{Red}};
%  outer
( 13.5,-36)*{\afacdmexsqcol{}{}{}{}{}{}{33}{4}{Red}{}{Red}};
( -2.5,-36)*{\afacdmexsqcol{}{}{}{}{}{}{33}{4}{Red}{}{Red}};
}\quad,
\]
where $A_1,A_2,A_3$ are isomorphic to $A$ and $B_1,B_2,B_3$ are isomorphic to $B$.

With the assistance of the atomic flow observe the following:
\begin{itemize}
\item If $\epsilon$ is an edge connecting two of $A_1,A_2,A_3,A_4,B_1,B_2,B_3$ or $B_4$ then $a_k$ or $\bar a_k$ maps to $\epsilon$, so since $\Norm(\Core(\Phi),a_1,\dots,a_{k-1})$ is weakly streamlined and contains no (co)interaction instance where $a_k$ or $\bar a_k$ occurs, there is no edge $\epsilon'$ in a path from an interaction vertex to a cointeraction vertex, such that one of $a_1,\dots,a_{n-1}$ maps to $\epsilon'$.
\item None of the edges that could be in a path from the interaction vertex (red) coincide with any of the edges that could be in a path from the cointeraction vertex (green).
\end{itemize}

Since there are no paths from interaction vertices to cointeraction vertices, $\Norm(\Core(\Phi),a_1,\dots,a_k)$ is weakly streamlined. Furthermore, the only atoms which occur in (co)interaction instances in $\Norm(\Core(\Phi),a_1,\dots,a_k)$ which did not in $\Norm(\Core(\Phi),a_1,\dots,a_{k-1})$ are $a_k$ and $\bar a_k$.
\end{proof}

%===========================================

\section{Conclusion}


\bibliographystyle{alpha}
\bibliography{di-biblio}

\end{document}